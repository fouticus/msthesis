\chapter{Future Work}
\label{chap:future}

methods that use more data: unsupervised pretraining, CAPRI, 

\section{Data}

\section{Unsupervised Pretraining}

use pretraining

convolutional autoencoder
- deconvolution involves transpose weight matrices. 
	- encode edges to generate a representation. This admits deconvolutions

\subsection{Alternative Data Sources}

Redundant training data set (with "fair" sampling)

CAPRI, other data sets...


\section{Model}

\subsection{Simplicial Complex Convolution}

- concept of edge coupling was to differentiate different neighbors, so not all the same. 
	this accounts for relationship of neighbors to center vertex. 
- In image convolutions, groups of "neighbor" pixels are close to one another, so that collectively they may indicate the presence of something interesting in a part of the receptive field. 
	this is referring to the relationships of neighbors to one another. 

	To capture this we can create a simplicial complex based on some threshold, average activations over vertices for each simplex in the neighborhood, and then max over all simplices to capture interesting "regions" of the receptive field. 
		- generalization of graph convolution.


\subsection{RFPP Optimization}

optimize RFPP with different loss function or method.

\subsection{Ensembles}

- Bagging ensemble to train domain experts.
- Different depths to allow multi-resolution approach.


\subsection{Multi-coupling}

- capture advantages of sum and product coupling by doing both.


\section{Other Problems}

QSAR


