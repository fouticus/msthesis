%%%%%%%%%%%%%%%%%%%%%%%%%%%%%%%%%%%%%%%%%%%%%%%%%%%%%%%%%%%%%%%%%%%%%%
% Colorado State University LaTeX Thesis Template and Documentation
%
% by
%   Elliott Forney
%
% This is free and unencumbered software released into the public domain.
% 
% Anyone is free to copy, modify, publish, use, compile, sell, or
% distribute this software, either in source code form or as a compiled
% binary, for any purpose, commercial or non-commercial, and by any
% means.
% 
% In jurisdictions that recognize copyright laws, the author or authors
% of this software dedicate any and all copyright interest in the
% software to the public domain. We make this dedication for the benefit
% of the public at large and to the detriment of our heirs and
% successors. We intend this dedication to be an overt act of
% relinquishment in perpetuity of all present and future rights to this
% software under copyright law.
% 
% THE SOFTWARE IS PROVIDED "AS IS", WITHOUT WARRANTY OF ANY KIND,
% EXPRESS OR IMPLIED, INCLUDING BUT NOT LIMITED TO THE WARRANTIES OF
% MERCHANTABILITY, FITNESS FOR A PARTICULAR PURPOSE AND NONINFRINGEMENT.
% IN NO EVENT SHALL THE AUTHORS BE LIABLE FOR ANY CLAIM, DAMAGES OR
% OTHER LIABILITY, WHETHER IN AN ACTION OF CONTRACT, TORT OR OTHERWISE,
% ARISING FROM, OUT OF OR IN CONNECTION WITH THE SOFTWARE OR THE USE OR
% OTHER DEALINGS IN THE SOFTWARE.
%%%%%%%%%%%%%%%%%%%%%%%%%%%%%%%%%%%%%%%%%%%%%%%%%%%%%%%%%%%%%%%%%%%%%%

% Preamble
%%%%%%%%%%%%%%%%%%%%%%%%%%%%%%%%%%%%%%%%%%%%%%%%%%%%%%%%%%%%%%%%

% use the thesis document class
% this is derived from the standard book class
% and supports many of the same features
\documentclass[master]{thesis}
%\documentclass[doctor]{thesis} % for a dissertation

% fonts
% use times font by default but can specify other fonts
%\usepackage{fourier} % fourier is also a nice choice

%\usepackage[scaled]{helvet} % or these two lines give a sans-serif font
%\renewcommand\familydefault{\sfdefault} 

% this is useful for including dummy test
% it can be removed in final document
\usepackage{lipsum}

% ams math packages
\usepackage[cmex10]{amsmath}
\usepackage{amsthm,amssymb}

% graphics packages
\usepackage[pdftex]{graphicx} % remove pdftex if you are not compiling to pdf
\graphicspath{{./figures/}} % this places all graphics in the figures subdirectory

% allowed graphics extensions
% uncomment if you prefer to add extension in \includegraphics
\DeclareGraphicsExtensions{.pdf,.png,.jpg}

% allows the creation of subfigures
\usepackage[caption=false]{subfig}

% book tables are simple and look nice
\usepackage{booktabs}

% for specifying urls and links
\usepackage{url}
\urlstyle{same} % same style as regular text

% Fout added stuff
\usepackage{xcolor}
\usepackage{soul}
\usepackage{multirow}
\newtheorem{definition}{Definition}[section]



% Title Page
%%%%%%%%%%%%%%%%%%%%%%%%%%%%%%%%%%%%%%%%%%%%%%%%%%%%%%%%%%%%%%%%

% title of your thesis
\title{Protein Interface Prediction Using Graph Convolutional Networks}

% author's name
\author{Alex M. Fout}

% author's email address
\email{fout@colostate.edu}

% department name
\department{Department of Computer Science}

% semester of completion
\semester{Fall 2017}

% committee member names
\advisor{Asa Ben-Hur}
\committee{Chuck Anderson} % as many committee entries as you need
\committee{Hamid Reza Chitsaz}
\committee{Wen Zhou}

% Copyright Page
%%%%%%%%%%%%%%%%%%%%%%%%%%%%%%%%%%%%%%%%%%%%%%%%%%%%%%%%%%%%%%%%

% here is an example of student copyright declaration
% note that the \copyright command prints the copyright symbol,
% so we use the name \mycopyright instead
\mycopyright{%
Copyright by Alex M. Fout 2017 \\
All Rights reserved
}

% here is an example of a creative commons copyright license
% ask the graduate school for more information
%\mycopyright{%
%This work is licensed under the Creative Commons Attribution-NonCommercial-NoDerivatives 3.0 United States License.
%
%\vspace{3em}
%
%To view a copy of this license, visit:
%
%\vspace{2em}
%
%\url{http://creativecommons.org/licenses/by-nc-nd/3.0/legalcode}
%
%\vspace{3em}
%
%Or send a letter to:
%
%\vspace{2em}
%
%Creative Commons
%
%171 Second Street, Suite 300
%
%San Francisco, California, 94105, USA.
%}

% Abstract
%%%%%%%%%%%%%%%%%%%%%%%%%%%%%%%%%%%%%%%%%%%%%%%%%%%%%%%%%%%%%%%%

\abstract{%
Proteins play a critical role in processes both within and between cells through their interactions with each other and other molecules.
Proteins interact via an \emph{interface} forming a complex, which is difficult, expensive, and time consuming to determine experimentally, giving rise to computational approaches.
These computational approaches utilize known electrochemical properties of protein amino acid residues in order to predict if they are a part of an interface or not.
Prediction can occur in a partner independent fashion, where amino acid residues are considered independtly of their putative neighbor, or in a partner specific fashion, where pairs of potentially interacting residues are considered together.
Ultimately, prediction of protein interfaces can aide in illuminate cellular biology, improve our understanding of diseases, and aide pharmaceutical research.
Interface prediciton has historically been performed with a variety of methods, to include docking, scoring functions, template matching, and more recently, machine learning approaches.

The field of machine learning has undergone a revolution of sorts with the emergence of \emph{convolutional neural networks} as the leading method of choice for a wide swath of tasks.
Enabled by massive quantities of data and the increasing power and availability of computing resources, convolutional neural networks efficiently detect patterns in grid structured data and generate hierarchical representations that prove useful for many types of problems.
This success has motivated the work presented in this thesis, which seeks to improve upon state of the art interface prediction methods by incorporating concepts from convolutional neural networks.

Proteins are inherently irregular, so they don't easily conform to a grid structure, whereas a graph representation is much more natural.
Various convolution operations have been proposed for graph data, each geared towards a particular application.
We adapt these convolutions and propose two new variants, and perform partner specific prediction by training on the Docking Benchmark Dataset version 4.0 complexes and testing on the new complexes added in version 5.0.
We also compare against the state of the art method partner specific method, PAIRpred~\cite{minhas2014}.
Results show that multiple variants of graph convolution outperform PAIRpred, with no method emerging as the clear winner.

In the future, additional training data may be incorporated from other sources, unsupervised pretraining such as autoencoding may be employed, and a generalization of convolution to simplicial complexes may also be explored.
In addition, the various graph convolution approaches may be applied to other applications with graph structured data, such as Quantitative Structure Activity Relationship (QSAR) learning, and knowledge base inference.

}

% Acknowledgments 
%%%%%%%%%%%%%%%%%%%%%%%%%%%%%%%%%%%%%%%%%%%%%%%%%%%%%%%%%%%%%%%%

\acknowledgements{%
I would like to firstly acknowledge and thank my Creator and Savior Jesus Christ, who continues to bless me with new challenges and show me the wonder of His creation.
Thank you to my parents for their unconditional love and support that has manifested itself in so many practical ways, and to Kris, Katherine, Steve, Quinn, and members of my extended family for their support and understanding as I've worked towards completion of my degree.
My immense grattitude goes to my advisor, Dr. Asa Ben-Hur for his patience, calm guidance, and encouragement to maintain a healthy life balance in graduate school of all places.
I thank the members of my committee for lending their expertise to provide feedback and insight on the work presented in this thesis, to include Chuck Anderson, Hamid Reza Chitsaz, and Wen Zhou.
I must thank my fellow interface predictors, Basir Shariat and Jonathon Byrd, who very much compensated for my shortcomings in numerous ways, and who bravely held me back from the precipice of software engineering.
We make a good team!
I appreciate all of the support and feedback from the other members of Asa's research group, to include (alphabetically) Gareth Halladay, Mike Hamilton, Don Neumann, Swapnil Sneham, and Fahad Ullah.


Some molecular graphics and analyses were performed with the UCSF Chimera package. Chimera is developed by the Resource for Biocomputing, Visualization, and Informatics at the University of California, San Francisco (supported by NIGMS P41-GM103311).


\vfill
\noindent
Soli Deo Gloria
}

% Metadata
%%%%%%%%%%%%%%%%%%%%%%%%%%%%%%%%%%%%%%%%%%%%%%%%%%%%%%%%%%%%%%%%%%%%%%

% consider using hyperref to insert pdf metadata and make links clickable
% safe to remove if not using pdf or if it causes problems
\usepackage[pdfpagelabels,pdfusetitle,colorlinks=false,pdfborder={0 0 0}]{hyperref}

\begin{document} % preamble is complete, add any custom packages above
%%%%%%%%%%%%%%%%%%%%%%%%%%%%%%%%%%%%%%%%%%%%%%%%%%%%%%%%%%%%%%%%

\frontmatter % starts preliminary pages
%%%%%%%%%%%%%%%%%%%%%%%%%%%%%%%%%%%%%%%%%%%%%%%%%%%%%%%%%%%%%%%%

\maketitle
\makemycopyright
\makeabstract
\makeacknowledgements


\tableofcontents
\listoftables % optional
\listoffigures % optional

\mainmatter % starts thesis body
%%%%%%%%%%%%%%%%%%%%%%%%%%%%%%%%%%%%%%%%%%%%%%%%%%%%%%%%%%%%%%%%



%%%%%%%%%%%%%%%%%%%%%%%%%%%%%%%%%%%%%%%%%%%%%%%%%%%%%%%%%%%%%%%%

\chapter{Introduction}
\label{chap:intro}

Many cellular processes rely on proteins, which facilitate these processes via their interactions with one another and other elements within the cell. 
%TODO: mention protein interaction networks?
A more complete understanding of how proteins interact with one another can provide insight into certain diseases, aide pharmaceutical research~\cite{fauman2003}, and improve our understanding of complex cellular processes~\cite{altman2003}.
Proteins normally interact with one another in pairs, via an \textit{interface} which is comprised of amino acid residues in each protein which participate in the interaction.
Experimentally identifying the interface between two proteins is a time consuming and expensive process which involves crystallization of the protein complex, imaging via x-ray crystallography or nuclear magnetic resonance, and sequence-structure alignment. 
\textit{In silico} methods to predict the 3D structure given information about each constituent protein, may detect unforeseen interfaces and help inform the potential relevance of various wet lab experiments.
Such methods exist, but are often based on energy minimization techniques which are computationally expensive \cite{esmaielbeiki2015}.
Some of these methods (Haddock~\cite{zundert2016} for example) allow the user to suggest amino acid residues which are likely part of the interface, to help bias the algorithm towards a correct solution.
This has motivated work to predict amino acid pairs which constitute part of the interface without solving for the entire 3D structure at once. 

This paper presents a novel method of predicting which amino acid pairs are part of the protein interface, which is inspired by the success of convolutional neural networks in image processing. 
We represent proteins as graphs and define a convolution operation which receptive field is a local connected subgraph of the protein graph.
Our neural networks convolve over each protein and make predictions on pairs of amino acids of the likelihood that they constitute a part of the interface. 
Our method outperforms an existing approach based on a support vector machine which uses pairwise kernels~\cite{minhas2014}.
The remainder of the introduction describes proteins and protein interface predtiction in more detail, and describes artificial neural networks in general, and convolutional neural networks specificially, as a primer for the graph convolutional networks presented in this paper.

Chapter \ref{chap:methods} describes our graph convolutional networks and the deep learning architecture we used for interface prediction. Chapter \ref{chap:experiments} describes the data set and experiments performed and presents our findings. Chapter \ref{chap:future} lays out potential avenues of research which build upon the findings in this paper. Appendix \ref{appendix:features} contains details pertaining to the features computed for each amino acid residue, and Appendix \ref{appendix:tools} describes the software and tools that were created and/or used for this research.

\section{Proteins}

DNA is often considered the "blueprint for life". 
In that case, proteins are the physical realization of those blueprints.
In other words, the information contained in DNA is ultimately used to synthesize proteins.
Proteins are composed of amino acids linked in a chain and held together by covalent bonds.
Amino acids are organic compounds consisting of a central "$\alpha$-carbon" atom, which bonds to an amine group ($NH_2$), a carboxyl group ($COOH$), a single hydrogen atom, and a so-called side chain, in a tetrahedral geometry.

\begin{figure}
	\centering
	%\begin{center}
	\includegraphics[width=0.5\textwidth]{wiley_liss_structural_bioinformatics_amino_acids.png}
	%\end{center}
	\caption{\small Graph convolution on protein structures.  Left:  Each residue in a protein is a node in a graph where the neighborhood of a node is the set of neighboring nodes in the protein structure; each node has features computed from its amino acid sequence and structure, and edges have features describing the relative distance and angles between residues.  Right:  Schematic description of the convolution operator which has as its receptive field a set of neighboring residues, and produces an activation which is associated with its center residue.}
	\label{fig:graph_rep}
\end{figure}

%TODO:figure?
There are 20 different types of amino acid, each of which has a different side chain structure which gives rise to distinct structural and electro-chemical properties. (table?)

Amino acids link together when the nitrogen atom from one protein's amine group bonds covalently with the carbon atom of another protein's carboxyl group, releasing a water molecule in the process.
This covalent bond is called a peptide bond, and an amino acid involved in at least one such bond is referred to as an \textit{amino acid residue}, or \textit{residue}.

A \textit{peptide}, or \textit{peptide chain}, is a linear chain of amino acids held together by peptide bonds, and the \textit{backbone} of the peptide consists of all atoms participating in peptide bonds together with the $\alpha$-carbons.
If a peptide contains several residues it is referred to as a \textit{polypeptide}.
Proteins consist of one or more polypeptide which are bound together.
All peptides have a natural ordering of their amino acid residues defined by the order in which they were incorporated into the peptide during protein synthesis. (is this true?)
The first residue to be incorporated has a free amine group(true?) denoted the \textit{N-terminus}, whereas the last residue has a free carboxyl group(true?) denoted the \textit{C-terminus}.


A residue's side chain can influence how it interacts with other amino acids or other atoms and molecules. For example, oppositely charged side chains will be attracted to each other, and polar and non-polar side chains will not strongly interact with each other. 
polar side chains are also called hydrophilic due to their affinity for water, and non-polar side chains are also called hydrophobic for the opposite reason.
Residues in the same protein, by virtue of their spatial proximity, will tend to interact with one another. 
This influences the folding of the protein into a 3d structure.

%TODO: describe domains or motifs?
%TODO: mention globular proteins vs other kinds?


\subsection{Protein Structure}

Protein structure can be described via four different layers of abstraction, each building upon the previous layer. 
\textit{Primary structure} refers to the sequence of amino acid residues (from N-terminus to C-terminus) in a single polypeptide, and is determined by the sequence of codons in the corresponding coding mRNA from which the protein is translated.
Sequences are typically written using a string of letters, where each unique letter corresponds to a different residue type. 

The physical chemistry associated with peptide bonds gives rise to the property that the Nitrogen and Carbon atoms involved in the peptide bond, along with the adjacent $\alpha$-carbons, all lie within a plane, denoted the \textit{amide plane}.
Each $\alpha$-carbon lies on the intersection between two amide planes, and the planes are free to rotate with respect to each other. 
Their relative orientation is described by two angles, $\psi$ and $\phi$, which describe the rotation of each plane with respect to the $\alpha$-carbon's tetrahedral geometry. (figure?)
In some cases, side chains prohibit certain angles due to \textit{steric constraints}, which enforce that no two atoms may occupy the same volume of space at the same time.
The angles $\phi$ and $\psi$ allow a certain amount of flexibility in the peptide backbone, which gives rise to higher order structures.

\textit{Secondary structure} describes the common local structures that arise from the interaction of non-adjacent residues.
There are three such categories of local structures: helices, sheets, and loops.
A helix occurs when the polypeptide coils into a barrel-like structure (like the threads of a screw) and residues from adjacent coils form hydrogen bonds with one another.
There are multiple types of helices, each determined by the sequence-distance between to hydrogen bonded residues.
The most common type is the $\alpha$-helix, corresponding to a sequence distance of (???). (figure?)
A sheet occurs when two non-adjacent sections of the polypeptide align next to each other such that residues in one of the sections form hydrogen bonds with residues in the other section.
Sheets may be parallel, in which the upstream portion (nearer the N-terminus) of each section bond with each other as well as the downstream portions (nearer the C-terminus), or anti-parallel, in which the upstream portion of one section is bonds with the downstream portion of the other section, and vice versa. (figure?)
In either case, these are referred to as $\beta$-sheets. 
Some sections of the polypeptide may form neither helices nor sheets, and are sometimes called \textit{loops}.
These sections are more flexible than helices or sheets due the lack of hydrogen bonds, and therefore are useful in connecting the end of one helix/sheet to the beginning of another. 

Helices and sheets provide some rigidity to a polypeptide, but it typically further "folds" together. This resulting 3D global structure is considered the \textit{tertiary structure} of a protein. 

Finally, in some cases multiple polypeptides combine together into a complex.
Because a polypeptide complex may perform a singular function that is not achievable by each of its constituent chains, it is sometimes referred to as a single protein.  (true?)
In this case, the \textit{quaternary structure} describes the manner in which the individual polypeptides combine together to form the complex. 



\subsection{Protein Interfaces and their Prediction}

The locus of protein interaction is the interface, which is comprised of pairs of residues, one from each interacting protein, which interact with one another and bind the two proteins together.
Such interaction between residues requires electrochemical and geometric compatibility in the local neighborhood of a residue pair.
For example, like-charged residues will be counter-conducive to the formation of an interface due to mutual repulsion, and opposite-charged residues will have the opposite effect.
Polar residues are more likely to interact with other polar molecules than with non-polar molecules, and non-polar residues are more likely to interact with other non-polar molecules than polar ones.
The notion of geometric compatibility suggests that portions of the interface on each protein should exhibit some form of shape complementarity with each other. 
Because these factors are important in the formation of interfaces, they are also relevant to the prediction of interfaces.

Historically, there have been two variants of the protein interface prediction problem: \textit{partner independent} and \textit{partner specific}.
The former variant considers a single residue from a single protein and attempts to answer the question: does this residue form a part of an interface with any protein? (check this)
The latter variant considers pairs of residues, each from a different protein, and attempts to answer a more specific question: does this \textit{pair} of residues constitute part of an interface between these two specific proteins. 
The pairwise nature of partner specific prediction poses technical challenges when developing methods to answer the research question, but considering a residue pair instead of a single residue allows the consideration of the compatibility of the pair.
Partner specific prediction is also more relevant to a biologist, because it identifies a correspondence between the two proteins, rather than considering one protein at a time.

Interfaces exhibit a local spatial correlation in that a residue is more likely to be a part of an interface if nearby residues are also a part of the interface. 
Local spatial correlation can aide prediction if information from nearby residues is taken into account during prediction.

In some cases, proteins undergo conformational change when binding because the presence of another protein changes the most energetically favorable state. 
In fact, sometimes this conformational change can actually facilitate the function of a protein or complex.
This can hinder prediction since the bound conformation is not known a priori and so the prediction must be made on the basis of the unbound protein structures. 

Methods of interface prediction include \textit{docking methods}(cite), \textit{template-based methods}(cite), and \textit{machine learning methods}.

Docking methods predict the 3D bound formation of two proteins in a complex and extract the interface from the predicted formation. 
These methods typically involve modeling the protein system electrical potential(check?) caused by each atom in the proteins, and performing an optimization algorithm to identify a configuration which minimizes the energy of the system. (more detail here and fact check)
Unlike template and machine learning methods, docking methods do not require a library of known interfaces in order to make predictions.

Template based methods make predictions based on similarity to a known template protein complex. 
The protein of interest is compared to a library of proteins whose interface is known, and the interface for the protein of interest is predicted using the interfaces of similar proteins (more detail here and fact check).

Machine learning methods attempt to directly predict the interface rather than comparing against a template complex or predicting the bound formation, however they still use information from a library of complexes whose interfaces are known.
Machine learning approaches have included use of a neural network (cite) and a support vector machine(cite). (more detail here and fact check)
The SVM based approach is called PArtner-specific Interacting Residue PREDictor (PAIRpred) (cite), and utilizes multiple radial basis function kernels. 
PAIRpred incorporates both sequence and structural information of residues, and uses pairwise kernels which operate on pairwise residues.
The method performs well compared to existing docking and machine learning methods (how about template based methods?).

%TODO: Talk about neural network approach


The work described in this paper performs partner specific interface prediction using a more sophisticated neural network architecture than previous methods. 


\section{Artificial Neural Networks}

\textit{Feed forward artificial neural networks}, or less formally, \textit{neural networks}, are a class of machine learning algorithm which are loosely inspired by neuroscientific models of how biological neural networks operate. 
Conceptually, a (artificial) neural network represents a parameterized function that is applied to an input data vector which produces an output, which may be a vector or scalar. 
For example, the input vector may be pixel values in an image of interest, and the output function may be a label applied to that image that indicates its content.

(equation)

This function can be decomposed into a sequence of \textit{layers}, each of which performs a relatively simple mathematical operation on its input to produce a an output.
The first layer is taken to be the input data, and the output of the last layer is taken to be the output of the network. 
All intermediate layers are termed \textit{hidden layers}, each of which accepting the output of the prior layer, performing a mathematical operation, and passing the output to the subsequent layer.
The number of layers in a neural network is taken to be its \textit{depth}.
The operations at each layer commonly take the form of a matrix multiplication, where a matrix of weights is multiplied by the input vector to produce an output vector.
This is called a \textit{dense layer}, and mathematically takes the form:

(equation). (include period)

The number of outputs of a given layer is also denoted the number of \textit{units} (or "neurons") in that layer.
This interpretation indicates that each unit in a layer takes a weighted sum of the inputs (or outputs from the previous layer), with each unit having a different set of weights for each input. 
It is also common to include a bias term in each layer and to apply some sort of nonlinear function elementwise on the weighted sum, and the result is considered the \textit{signal} of the layer.
Despite being conceptually simple, it has been proven that this formulation of artificial neural networks can approximate any function, provided it has enough units (look this up, check details and cite).

The challenge of using neural networks for function approximation is in finding the appropriate set of parameters to approximate the desired function. 
For feed forward neural networks, this is accomplished through a supervised training algorithm called backpropagation. 

%TODO: write about backpropagation
%TODO: write about overfitting, etc?

In recent years (cite), more advanced methods for neural networks have demonstrated success in sophisticated tasks, particularly for image related tasks (cite). 
These advances were catalyzed by the availability of large volumes of labeled image data and the improvements in computing power from, for example, general purpose graphical processing units (GP-GPUs) and distributed systems.
These factors allowed experimentation with larger networks and more complicated layer operations, to include convolutional layers.

%TODO: describe dense layers?

\subsection{Convolutional Neural Networks}

Traditionally, neural networks do not explicitly account for inherent structure in the inputs which may be useful when approximating the output function.
For example, pixels in an image are often spatially correlated, but spatial information is not explicitly given to the network.
Convolution layers are one way to incorporate spatial information in a neural network. 
The name comes from an interpretation of how they operate on a set of inputs with a well defined grid structure, such as a pixelated image or discrete time series, where a filter of weights (e.g. 3x3 pixels for images) is convolved over the input grid.
That is, at each position, the filter weights are multiplied with the corresponding inputs, called the \textit{receptive field}, and summed to produce a scalar value.
As with simpler layers, the output is often passed through a nonlinear function
Because each filter multiplication is associated with a position of the input grid, the output retains a grid structure. 
In color images, each pixel has multiple values associated with it which capture color information, often referred to as \textit{channels}, and the filter typically has the same number of channels as the input. 
A convolutional layer may contain multiple filters, each of which produces a grid of scalar values which can be considered a channel in the output.
The weights in a filter determine the which inputs produce a larger signal and which do not, so a filter can be viewed as detecting certain patterns in the input.
An alternate interpretation of convolutional layers is that weights are being shared across a layer such that each unit in the convolutional layer receives inputs form a different localized region of the input, and that all units share weights. 

(equation)
(figure of convolution?)


Convolutional layers exhibit certain properties which aide in the classification tasks.
\textit{local pattern detection}: convolutional filters are usually significantly smaller than the size of the input, so for any given position of the filter they are operating on a local neighborhood of the input. 
\textit{global application}: since filters are convolved over the input, local patterns can be detected no matter where they occur in the input.
\textit{stackability}: the output of a neural network retains the same grid structure as the input, so convolutional layers can be stacked together.
The receptive field of a unit in a stacked layer consists of units in the prior convolutional layer, each of which has its own receptive field in the input.
Hence stacked layers have effectively larger receptive fields on the original input, and are therefore able to detect larger patterns in the original input which are typically more sophisticated.
Stacked convolutions on images have demonstrated an ability to learn a conceptual hierarchy of patterns in the input, from simple textures and edges to complex objects like faces or vehicles. 
\textit{robustness to overfitting}: the weight sharing inherent in convolutional layers means there are fewer trainable weights compared to a dense layer with the same number of hidden units. 
This reduces the propensity of the network to overfit to the training data and improves its ability to generalize to unseen examples.

Convolutional neural networks often incorporate \textit{pooling layers}, which reduce a small region of a grid (e.g. 2x2 pixels for an image) to a single grid element that takes either the maximum or average value (per channel) of the region. 
Pooling allows downsampling of the grid, and is often performed in between convolutional layers. 

The convolutions described above can only operate on a regular grid. 
Proteins, however, have inherently irregular structure. 
In this work we adapt the notion of convolution to work with irregular structures, which are represented as graphs. 







%TODO: How much to describe each feature? refer to appendix? what if not in appendix?

\chapter{Prior Work in Interface Prediction}
\label{chap:relatedwork} 

%In silico prediction of interfaces began in the 

As previously mentioned, the experimental determination of protein complexes is time and resource intensive, prompting computational modeling approaches.
Esmaielbeiki et al (2015)~\cite{esmaielbeiki2015} describe three slightly different computational problems: protein interaction prediction, protein interface prediction, and protein-protein docking~\cite{esmaielbeiki2015}.
The first problem seeks to identify pairs of proteins which interact, elucidating the complicated protein interaction networks that give rise to cellular processes. 
The second problem, and the focus of this thesis, is concerned with identifying the specific residues or pairs of residues which make up the interface.
The third problem considers two specified proteins and seeks the bound 3D structure of their complex.

\section{Docking}

Docking begins with the known unbound structures of two proteins known to interact, and conducts two main steps: search and scoring.
During search, the proteins are translated and rotated relative to each other and brought into contact to create a several putative 3D bound structures for the complex.
The putative structures are then evaluated by a scoring function to identify the most likely conformation of the complex.
Different docking methods differ in their the search algorithms and scoring functions.
Scoring functions may incorporate complementarity in geometry, chemistry, and electrostatics, or incorporate van der Walls forces or evidence based (aka statistical) potentials~\cite{tuncbag2011}\cite{janin1995}.
It is worth noting that docking methods can also be used to predict interfaces, by first solving for the 3D structure of the complex and then extracting the interface from the complex.
Indeed, docking methods were some of the earliest computational approaches developed for modeling protein interactions~\cite{janin1995}.
One of the major advantages of docking is its ability to produce interface predictions \textit{ab initio} without having examples of known interfaces, which is particularly useful when experimental complex data are sparse.
Unfortunately, docking methods traditionally suffer from relatively high false positive rates, are considerably less effective for complexes which undergo conformational change when binding than those that do not, and are computationally expensive because of the vast search space~\cite{janin1995}\cite{tuncbag2011}.

\section{Other Early Methods}

Some early alternatives to docking used sequence information, residue properties, and unbound structures for each protein in the complex to directly predict the interface without predicting the structure of the whole complex.
Lichtarge et al (1996)~\cite{lichtarge1996} used inferred evolutionary relationships between different proteins to identify conserved residues and then identified those conserved residues which lay on the surface of the protein.
This method was based on the hypothesis that conserved surface residues must be vital to a protein's function and therefore probably constitute an interface.
Pazos et al (1997)~\cite{pazos1997} took a similar approach, but instead looked at evolutionary relationships between protein complexes and identified pairs of residues between the proteins in the complex which have co-evolved.
This method requires only sequence information and therefore is applicable even in cases where the protein structures are unknown, but relies on having sufficient data to infer evolutionary relationships.
Additionally, this method is partner-specific since it identifies residue pairs which show correlated changes.
Gallet et al (2000)~\cite{gallet2000} used a sliding window on a protein sequence and calculated measures of hydrophobicity in a region, which can easily be calculated knowing the residue identities and secondary structure.
This method requires no phylogenetic information so is applicable even when no close evolutionary relatives can be identified.

Early methods such as those listed above were crafted for the available data and computational resources of the time, but were unable to fully account for the growing body of research surrounding protein interfaces and their properties, as in Jones \& Thornton, (1996)~\cite{jones1996}.
It was Jones \& Thornton, (1997)~\cite{jones1997} that proposed a method which incorporates multiple structural features such as surface planarity, protrusion, and accessible surface area, with residue level features such as solvation potential, hydrophobicity, and interface residue propensity.
They constructed a manual scoring function whose inputs are the aforementioned features and output is a score, where higher scores are intended to correspond with members of the interface.
They constructed a different scoring function for each of three different categories of complex, reflecting observations made into the characteristics of different complex types.
These categories were homomeric and small heteromeric proteins, larger heteromeric proteins, and antibody/antigen complexes.
Prediction was performed on small patches of residues.
Like docking, these methods avoid using examples of known interfaces when making predictions.

Evaluation and comparison of early methods was challenging due to the paucity of experimentally determined protein interfaces~\cite{esmaielbeiki2015}.
Thankfully, the turn of the twenty first century coincided with an increase in the number of experimentally determined structures added to databases such as the Protein Data Bank~\cite{berman2000}.
Curated subsets also emerged which focused on evaluating protein-protein docking methods, such as the Critical Assessment of Predicted Interactions (CAPRI)~\cite{janin2003} and the Docking Benchmark Dataset (DBD)~\cite{chen2002}.
These datasets also became useful in the evaluation (and sometimes training) of interface prediction methods.

\section{Data Driven Methods}
	
The increasing availability of data and increased interest in interface prediction led to a growing number and diversity of approaches.
Template based methods emerged which utilize a non-redundant library of known protein interfaces to make predictions about unknown proteins.
For a given query protein, a search is made in the library for known complexes where a partner is similar to the query protein.
The interface of the query protein is then inferred from the interfaces of the most similar query results.
Similarity may be measured via sequence or structural similarity~\cite{esmaielbeiki2015}.

Other data-driven methods have appeared which are based on either machine learning or statistical methods.
Some early machine learning based approaches used a support vector machine (SVM) to classify residues as either belonging to an interface or not.
%TODO: explain SVMs?
An SVM essentially provides a scoring function which is dependent on training data rather than manually constructed.
Koike \& Takagi, (2004)~\cite{koike2004} trained an SVM classifier to perform partner-independent prediction of interfaces.
They represented a residue by its profile, a vector of relative abundances of each amino acid type among homologous proteins at that location.
They experimented with different feature representations to make predictions at a particular residue, finding that incorporating profiles from sequential or spatially neighboring residues improves performance, as does incorporating accessible surface area and accounting for the relative interface size.
Bradford \& Westhead, (2004)~\cite{bradford2004} also performed partner-independent prediction with an SVM classifier, but instead made predictions for surface patches rather than individual residues.
Zhou \& Shan, (2001)~\cite{zhou2001} were among the first to train a neural network for partner-independent prediction.
They incorporated profile and solvent exposure of residues and their neighbors to make predictions at the residue level.
Their method uses residue profiles.
In a follow up paper, Chen \& Zhou, (2005)~\cite{chen2005} used an ensemble of neural networks to make a consensus prediction concerning a residue of interest.
%TODO: talk about proto-convolution of chen2005?

Various statistical approaches to interface prediction have also been proposed, many of which attempt to model the interdependence between different residues and between residue features.
Bradford et al, (2006)~\cite{bradford2006} compared a naive Bayes approach to a Bayesian network, which accounts for observed correlations between features, and found that both perform equivalently when predicting interface patches.
They also found that these methods perform well even when some data are missing, particularly when conservation scores can't be determined due the absence of homologs.
Friedrich et al, (2006)~\cite{friedrich2006} adapted a hidden Markov model (HMM) originally used for homology detection~\cite{eddy1998} in order to detect interacting residues.
The advantage of an HMM is the ability to jointly model all residues in a sequence at once.
Li et al, (2007)~\cite{li2007} generalized this joint modeling to an undirected graphical model using conditional random fields (CRFs) which performs comparably to other data based approaches.

Early machine learning and statistical methods for interface prediction provide predictions at the individual residue or patch level, in contrast to docking methods which generate global solutions for the complex.
These methods also typically incorporate both sequence and structural information in order to make partner-independent predictions.
However, in a 2007 review paper, Zhou \& Qin, (2007)\cite{zhou2007} identified the need for partner-specific methods in order to improve prediction specificity.


\section{Partner Specific Methods}

Whereas early partner-specific interface prediction methods are based on sequence co-evolution or derived from docking solutions, recent approaches have also included machine learning based methods.
Notably, two such methods have incorporated the same types of features as the partner-independent machine learning methods, but have instead considered pairs of residues from separate proteins when making predictions. 

In Prediciton of Protein-protein Interacting Position Pairs (PPiPP), Ahmad \& Mizuguchi, (2011)~\cite{ahmad2011} used a neural network which only uses sequence based features.
They experimented with two types of sequence based features, a sparse encoding and a position specific scoring matrix (PSSM) encoding.
%TODO: describe PSSM? refer to appendix?
The sparse encoding is a one-hot binary array of length 20 indicating the amino acid type.
The PSSM encoding instead represents each amino acid type by its log-odds frequency in iterative multiple-sequence-alignment results.
Using these features, they also experimented with different sized sequence-windows, where a residue of interest is represented by the concatenation of feature vectors of all residues inside a sequence window.
Both the feature representations and the sequence windows are in keeping with prior work in machine learning based partner-independent predictors.
Residues whose windows extended past the end of the bounds of the sequence were excluded from the training set.

The data used are from version 3.0 of the Docking Benchmark Dataset~\cite{hwang2008}, which includes 124 unbound and bound structures of both proteins.
A pair of residues, one from each protein is considered positive (part of the interface) if they are within 6\AA~of each other and negative otherwise.
Training examples were created by concatenating the feature representation of each residue in a pair together.
Due to the inherent asymmetry of this concatenation, two examples were produced for each pair by concatenating the representations in both orders (AB and BA).
Predictions for pairs were made by taking the average prediction of the two orderings.
There are significantly fewer positive than negative examples, so negative examples were sampled to prevent extreme class imbalance.
Specifically, either 2\% or 1000 negative examples were randomly selected, whichever was smaller.

The model consists of an ensemble of 24 neural networks, each using different window sizes for the sparse and PSSM encodings.
The predictions from each of these models are averaged to produce a final prediction for a residue pair.
Networks were evaluated in a leave-one-out fashion (train on all but one complex and test performance on the omitted complex).
Performance was measured using area under the receiver operating characteristic curve (AUC) for each left out complex and averaged.

The ensemble achieved an AUC of 72.9\% compared to 67.9\% for a single neural network with windows of size 7 for both encodings.
The authors compared this to an analogously constructed neural network ensemble which performs partner-independent prediction.
Examples consist of a single residue which is positive if it is part of an interface and negative otherwise.
Partner independent predictions were naively converted to partner-specific predictions by averaging the scores of each residue in the pair, yielding an AUC of 71.0\%, worse than the partner-specific predictor.
Conversely, partner-specific predictions can also be converted to partner-independent predictions by taking the max over all potential neighbors. 
This yielded an AUC of 66.1\% for the partner-independent prediction problem, better than the 63.8\% AUC of the partner-independent model.
Thus, the partner-specific model outperformed the partner-independent model on both partner-specific and partner-independent predictions.

In PAIRPred, Minhas et al (2014)~\cite{minhas2014} incorporated custom symmetric symmetric kernels into an SVM formulation~\cite{minhas2014}.
In addition, this method includes structural information for each residue as well.
The structure based features consist of the relative accessible surface area (rASA), residue depth, half sphere amino acid composition, and protrusion index.
The sequence based features include the same PSSM encoding as PPiPP, a position frequency scoring matrix (PSFM) encoding (like the PSSM encoding but with raw frequencies instead of log-odd frequencies), and predicted rASA (prASA).

The authors used complexes from DBD version 3.0~\cite{hwang2008} for comparison to PPiPP.
They also utilized the updated DBD version 4.0~\cite{hwang2010}, which is a superset of the complexes in version 3.0, and complexes from the CAPRI~\cite{janin2013} experiment.

The authors constructed specialized symmetric pairwise kernels which compute a similarity between any two pairs of residues, independent of the ordering of each pair.
Each pairwise kernel is constructed by taking a symmetric combination of kernels for individual residues, where residue kernels were themselves sums of radial basis function kernels for each feature type.
%TODO: explain this better and write out some math?
Several individual pairwise kernels are summed and the result is normalized to produce the final pairwise kernel.
The authors also investigated a postprocessing step where pair scores are smoothed based on the scores in a neighborhood around each residue.
Cross validation was used to tune the soft margin parameter \textit{C} and the residue kernels and pairwise kernels were optimized in similar fashion. 
Following the same leave-one-out procedure as Ahmad \& Mizuguchi, PAIRPred achieved an AUC of 87.3~\% before any postprocessing, and 88.7~\% after post processing.
When using only sequence based features, PAIRPred achieved an AUC of 80.9~\%, which demonstrates a significant improvement over PPiPP even in the absence of structural features.
Partner-independent prediction was performed by taking the maximum score across all potential neighbors, and achieved an AUC of 70.8~\% and 77.0~\% using only sequence features and all features respectively, which also outperforms PPiPP's best partner-independent performance.

Minhas et al (2004)~\cite{minhas2014} note that the extreme class imbalance inherent to the partner-specific classification problem means AUC is not as easy to interpret. 
Therefore they also calculate the rank of the first positive prediction (RFPP) at a given percentile, where $RFPP(p) = q$ means that p~\% of the complexes in the test set have at least one true positive pair among the top q predictions.
Low values of q corresponding to high values of p indicate better classifier performance because a higher percentage of complexes have a true positive near the top predictions.
The authors argue that this is more relevant to a biologist because it indicates the trustworthiness of the top few predictions of the classifier. 
PAIRPred outperforms PPiPP for values of p equal to 10, 25, 50, and 75, but is worse at the 90~\% level.

The authors conclude by noting that there is much room for improvement in partner-specific interface prediction, particularly for complexes with a high degree of conformational change. 
They propose adding new features to PAIRPred which capture shape complementarity between binding interfaces, co-evolution, and protein flexibility.
However, PAIRPred's improvement over PPiPP suggests that not only is the choice of features important to classifier performance, but also their representation and the construction of the model itself.
This supports the investigations of this thesis into the graph representation of proteins and new convolutional methods which operate on graphs.
Chapter \ref{chap:neuralnetworks} gives a primer on the convolutional neural networks which helped inspire these new methods.

%TODO: refer to grid convolutions as discrete (?)


\chapter{Artificial Neural Networks}
\label{chap:neuralnetworks}

\textit{Feed forward artificial neural networks}, or less formally, \textit{neural networks}, are a class of machine learning algorithm which are loosely inspired by neuroscientific models of how biological neural networks operate. 
In its most basic form, a neural network is a parameterized function which accepts a vector of data as input and produces either a scalar or vector output.
For example, the input vector may be pixel values in an image of interest, and the output function may be a label applied to that image that indicates its content.
In this thesis, the function is denoted $f(x|\Theta)$, where $x$ is the input vector and $\Theta$ is the complete set of parameters.
$f$ can be decomposed into a sequence of layers, $h_i(x_i|\Theta_i)$, which perform comparatively simple mathematical operations on their input to produce an output.
By convention, the input vector is considered the first layer of a neural network even though it performs no operations. 
The output of the last layer is the output of the network. 
All intermediate layers are termed \textit{hidden layers}, each of which accepts the output of the preceding layer, performs a mathematical operation, and passes the output to the subsequent layer.
The number of layers in a neural network (besides the first layer) is its \textit{depth}, and the number of outputs of a given layer is the number of \textit{units} (or \textit{neurons}) in that layer.
Figure \ref{fig:twolayernetwork} depicts a two layer neural network with three inputs, four units in the hidden layer, and two outputs. 

\begin{figure}
	\centering
	%\begin{center}
	\includegraphics[width=0.8\textwidth]{twolayernetwork.pdf}
	%\end{center}
	\caption{A two layer neural network with a single hidden layer.}
	\label{fig:twolayernetwork}
\end{figure}

A common form of hidden layer is a \textit{dense layer}, in which each unit calculates a weighted sum of the layer inputs, where the set of weights is unique to each unit.
It is also common to add a scalar bias term and apply a nonlinear activation function to the weighted sum.
Dense layers have a convenient mathematical representation.

\begin{equation}
h(x|W, b)=\sigma(W x^T + b),
\label{eq:denselayer}
\end{equation}

\noindent
where $W$ is a matrix of weights, $x$ is the layer input, $b$ is a vector of biases, and $\sigma(\cdot)$ is a nonlinear activation function.
If there are $n$ inputs to a layer and $m$ units in the layer, then $W\in\mathbb{R}^{nxm}$ and $b\in\mathbb{R}^{m}$.
The expression $W x^T + b$ is called the \textit{signal}, and $h$ the \textit{activation}.
For hidden layers, activations have traditionally taken the form of a sigmoid, or "s-shaped" function such as the logistic function $\sigma(x) = 1/(1+e^{-x})$ or hyperbolic tangent $\sigma(x) = tanh(x)$.

Despite being conceptually simple, this formulation of artificial neural networks is capable of approximating any continuous function on a compact subset in $\mathbb{R}$ with arbitrarily small error~\cite{cybenko1989}.
The challenge of using neural networks for function approximation is in finding the appropriate set of parameters to approximate the desired function.
For simple feed forward neural networks, this is accomplished by quantifying the error between the desired function and the approximation for given training data $X=(x_1, x_2, ..., x_N)$ with a differentiable loss function $L(\Theta | X)$. 
$f$ is updated by differentiating the loss function with respect to the network parameters $\Theta$ and "taking a step" in parameter space in the opposite direction of the gradient. 
This update step can be repeated iteratively in a process called \textit{gradient descent} and mathematically takes the form:
\begin{equation}
\Theta_{k+1} = \Theta_k - \eta \nabla L(\Theta_k | X) = \Theta_k - \eta \sum_{i=1}^{N} \nabla L(\Theta_k | x_n),
\label{eq:batch_gd}
\end{equation}

\noindent
where $\eta$ is a tunable step size and $k$ indicates the iteration.
Weights are usually initialized by drawing from distributions that have some empirical or theoretical justification~\cite{glorot2010}.
A variant of this algorithm, called \textit{stochastic gradient descent} performs an update using a gradient computed from a random subsample (without replacement) of the training data at each iteration.
When all training data has been sampled, this constitutes an \textit{epoch}.
%TODO: write about backpropagation? (and minibatching, etc...?)
Like many machine learning algorithms, neural networks risk \textit{overfitting}, in which the network learns to approximate the training data (which usually contain noise), rather than the underlying function from which the training data are assumed to be drawn.
This hinders the ability of neural networks to generalize to unseen data.

In recent years, more advanced forms of neural networks under the moniker \textit{deep learning} have demonstrated success in sophisticated tasks, particularly for those related to images~\cite{lecun2015}. 
These advances were catalyzed by the availability of large volumes of labeled image data and the improvements in computing power from general purpose graphical processing units (GP-GPUs) and distributed systems.
Such factors allow experimentation with larger networks and more complicated layer operations such as convolutions, which are described in the following section.

\subsection{Convolutional Neural Networks}

Traditionally, neural networks do not explicitly account for inherent structure in the input which may be useful when approximating the output function.
For example, pixels in an image are often spatially correlated, but spatial information is not explicitly given to the network when using dense layers.
Convolutional layers are one way to incorporate structure information in a neural network. 
The name comes from an interpretation of how they operate on a set of inputs with a well defined grid structure, such as a pixelated image or discrete time series, where a filter of weights (e.g. 3x3 pixels for images) is convolved over the input grid.
That is, at each filter position, the filter weights are multiplied with the corresponding inputs, called the \textit{receptive field}, and summed to produce a scalar value.
As with dense layers, the result is usually passed through an activation function.
Because each filter multiplication is associated with a position of the input grid, the output retains a grid structure.
Figure \ref{fig:convolutionallayer} depicts a simple convolutional layer.
In color images, each pixel has multiple values, or \textit{channels}, which capture color information, and the filter must have the same number of channels as the input in order to take the elementwise product between the filter and the receptive field.
A convolutional layer may contain multiple filters, each of which produces a grid of scalar values which can be considered a channel in the output.
The weights in a filter determine which receptive fields produce a larger output and which do not, so a filter can be viewed as detecting specific patterns in a receptive field.
An alternate interpretation of convolutional layers is that weights are being shared across a layer such that each unit in the convolutional layer receives inputs from a different localized region of the input, and that all units share weights appropriately.

%TODO: equation?

\begin{figure}
	\centering
	%\begin{center}
	\includegraphics[width=0.8\textwidth]{convolutionallayer.pdf}
	%\end{center}
	\caption{A convolutional layer with a single 3x3 filter operating on a grid input with a single channel. The filter is multiplied elementwise with the corresponding inputs and produces a scalar value.}
	\label{fig:convolutionallayer}
\end{figure}


Convolutional layers exhibit certain properties which aid in image related tasks:
\begin{itemize}
\item
\textit{Local pattern detection}: Convolutional filters are usually significantly smaller than the size of the input, so for any given position of the filter they are operating on a local neighborhood of the input. 
This means that patterns detected by a particular filter are local.
\item
\textit{Translational Invariance}: Since filters are convolved over the input, patterns can be efficiently detected regardless of where they occur in the input.
In a dense layer, detection of a local pattern in each part of the input requires that weights associated with each region of the input be trained individually to detect the same pattern.
This means the training data must represent the same pattern in all portions of the input. 
For example, to train a dense layer to detect a cat's ear regardless of where it appears in an image, it must be trained using data which represent a cat's ear in every possible position in the image.
\item
\textit{Stackability}: The output of a convolutional layer retains the same grid structure as the input, so the output of a convolutional layer can be used as input to another convolutional layer.
Consider a neural network with two convolutional layers stacked on top of one another.
The receptive field of a unit in the second layer consists of units in the preceeding convolutional layer, each of which has its own receptive field in the input.
Hence filters in the second layer detect patterns of patterns in the original input, which suggests that stacked layers learn a hierarchical representation of the original input.
Furthermore, stacked layers have effectively larger receptive fields on the original input, and are therefore able to detect larger patterns in the original input.
The ability to learn a conceptual hierarchy of patterns in the input, from simple textures and edges to complex objects like faces or vehicles, has been demonstrated empirically as well~\cite{zeiler2013}.
\item
\textit{Robustness to overfitting}: The weight sharing inherent in convolutional layers means there are fewer trainable weights compared to a dense layer with the same number of hidden units. 
This reduces the propensity of the network to overfit to the training data and improves its ability to generalize to unseen examples.
\end{itemize}

Convolutional neural networks often incorporate \textit{pooling layers}, which reduce a small region of a grid (e.g. 2x2 pixels for an image) to a single grid element that takes either the maximum or average value (per channel) of the region. 
Pooling allows downsampling of the grid, and is often performed between convolutional layers. 

The convolutions described above can only operate on a regular grid. 
Proteins, however, have inherently irregular structure. 
In this work the notion of convolution to work with irregular structure, which are represented as graphs, as described in Chapter \ref{chap:methods}.







\chapter{Methods}
\label{chap:methods}

The methods presented in this thesis were drawn from a desire to exploit local structure around a residue when performing interface prediction.
The biological motivation for this is that a residue's neighborhood influences its propensity to participate in an interface.
It was noted in Chapter \ref{chap:neuralnetworks} that convolutional neural networks are one way of detecting features in a local neighborhood, but they are limited to regular grids. 
Unfortunately, proteins cannot naturally be represented as a regular grid, so convolutions must be developed for a more natural representation: graphs.


\section{Proteins As Graphs}

An undirected, unweighted graph $G$ consists of a set of vertices, $V=\{v_1, v_2, ..., v_n\}$, and a set of eges, $E=\{e_1, e_2, ..., e_m\}$ where each edge is incident to two vertices and there is at most one edge between two vertices.
In a protein graph, each vertex represents a residue in the protein, and each edge represents the relationship between two residues.
Thus any information pertaining to a particular residue can be associated with the relevant vertex in the form of a feature vector.
The features used in this work are drawn from features used in prior interface prediction work \cite{minhas2014}.
Likewise, any information about the relationship between two residues can be associated with the relevant edge.
The edge features used here describe the distance between and relative orientation of two residues.
These edge features are defined between any two residues in the protein, so the graph is complete. 
The number of edge features and number of vertex features may not be the same.
A detailed explanation of each feature is contained in Appendix \ref{appendix:features}

This representation is a structural abstraction of the original protein to a well studied mathematical object.
It does not rely on a coordinate system, as is the case when working with raw 3D positions.
This makes biological sense because proteins often have no natural orientation in the cell, and the relative orientations of two interacting proteins is not known a priori.
However, the graph retains 3D structural information in the edge features, and is embedded in an underlying metric space.
This embedding is useful when defining local neighborhoods around vertices, which is necessary when designing convolutions.
With proteins represented as graphs, the remaining task is to design a convolution operation which operates on graphs. 

\section{Graph Convolutions}
Though the formulation and application of graph convolutions presented in this thesis are new, graph convolutions already exist in the literature.


\subsection{Prior Work In Graph Convolutions}
Recent years have seen increased attention for problems involving graph structured data, prompting developments in graph convolutions to perform various tasks on those data~\cite{bronstein2016}.
These approaches have generally fallen into two categories, \textit{spectral} and \textit{spatial}.

Spectral methods are based on linear functions in the "frequency domain" of a graph, defined using the laplacian operator $\mathcal{L}=I-D^{-1/2}WD^{-1/2}$, where $I$ is the identity matrix, $W$ is a similarity matrix (containing edge weights), and $D$ is a diagonal matrix containing the degree of each vertex~\cite{bruna2013, henaff2015, kipf2016}.
%TODO: mention(?): scaling to large graphs difficult because factoring large matrices difficult
Each filter in a spectral convolution implies a weighting of each frequency in the spectral decomposition of the graph~\cite{mallat2009}.

Spatial methods instead define operations in a localized neighborhood of a central vertex~\cite{henaff2015, atwood2016diffusion}.
Each neighborhood constitutes a receptive field where a convolution operation is performed. 
Convolutions commonly involve a vector of weights and take a weighted sum of neighbors, much like a discrete convolution on a grid can be viewed as taking a weighted sum of grid elements within the receptive field.
%TODO: cite papers with different neighborhood convolutions (write more about them?)
Spatial convolutions are more directly analogous to grid based convolutions as described in Chapter \ref{chap:neuralnetworks}, but introduce a problem of correspondence when translated to graphs.

\subsection{The Problem of Receptive Field Correspondence}
When convolving on a grid, each receptive field has an identical structure (for example 3x3 pixels in an image), so there is an automatic correspondence between receptive fields, such  that the same weights are applied to corresponding portions of all receptive fields. 
For example, the upper left pixel in a 3x3 receptive field is always multiplied by the same weight when taking the weighted sum, regardless of which receptive field is being considered.
With graphs, there is often no such correspondence from one receptive field to another (there is no "upper left" vertex in a vertex neighborhood), aside from the natural correspondence of central residues. Hence the issue is what to do with the neighbors. To complicate things, the number of neighbors in a receptive field may not be constant and is dependent on how the receptive field is defined.
This problem of correspondence has traditionally been addressed with two approaches that are summarized below. 
\begin{enumerate}
	\item \textit{Imposed ordering of neighbors}. This approach generates a correspondence between two receptive fields by ordering the neighbors in each and associating neighbors of a common position. 
	Ordering methods can be based on vertex characteristics, like degree and betweenness centrality, or some domain specific knowledge ~\cite{niepert2016, duvenaud2015}.
	They also typically require the number of neighbors in a receptive field to remain the same.
	This approach allows filter weights which are tied to a particular position in the ordering, which, it is assumed, has some invariant significance across all receptive fields.
	However, the imposed ordering is often arbitrary, limiting the usefulness of this approach.
	
	\item \textit{Identical treatment of neighbors}. This approach ignores the need to establish a correspondence between receptive fields and instead treats all neighbors identically.
	Rather than applying different weights to neighbors depending on their position in an ordering, the same weights are applied to each neighbor.
	This allows for different sizes of receptive fields and avoids choosing an ordering method, but lacks the ability to treat neighbors differently based on their relationships to each other and to the central vertex.
	However, this method fails to capture the possible unique structural relationship between each neighbor and the central vertex.
\end{enumerate}
Figure \ref{fig:correspondence_approaches} shows a graphical depiction of both approaches.

\begin{figure}
	\centering
	%\begin{center}
	\includegraphics[width=0.8\textwidth]{correspondence_approaches.pdf}
	%\end{center}
	\caption{Two approaches of establishing correspondence between the neighbors of receptive fields A and B. Central vertices are shown in blue and neighbors in green. The central vertices always correspond with one another. Left: neighbors are ordered and associated based on position. Unique weights (\textit{$w_2$--$w_4$}) can then be applied to each position in the order. Right: neighbors are left unordered and treated identically. This requires that the same weights (\textit{$w_2$}) be used for all neighbors.}
	\label{fig:correspondence_approaches}
\end{figure}


\subsection{Order Free Coupled Graph Convolutions}
This thesis presents a graph convolution which avoids imposing an arbitrary ordering on the neighbors in a receptive field but also avoids treating all neighbors the same.
This is accomplished by incorporating information from the edges between each neighbor and the central vertex.
There are two variants of graph convolution which differ in how the edge information is incorporated, denoted \textit{sum coupled} and \textit{product coupled}.
For a central vertex $i$ on the graph and a local neighborhood of vertices $\mathcal{N}_i$, the output of sum coupled graph convolution is:
\begin{equation}
h_i(x | W^\textsc{c}, W^\textsc{n}, W^\textsc{e}, b) = \sigma \bigg( W^{\textsc{c}} x_i + \frac{1}{|\mathcal{N}_i|}\sum_{j \in \mathcal{N}_i} (W^{\textsc{n}} x_j + W^{\textsc{e}} A_{ij}) + b \bigg),
\label{eq:sum_coupling}
\end{equation}
where $x_i$ is the feature vector associated with vertex $i$, $A_{ij}$ is the feature vector associated with edge $(i, j)$, $W^\textsc{c}$, $W^\textsc{n}$ and $W^\textsc{e}$ are weight matrices, and $b$ is a vector of biases. 
If there are $l$ vertex channels, $p$ edge channels, and $k$ filters, then $W^\textsc{c}\in\mathbb{R}^{k \times l}$, $W^\textsc{n}\in\mathbb{R}^{k \times l}$, $W^\textsc{e}\in\mathbb{R}^{k \times p}$, and $b\in\mathbb{R}^{k}$.
Intuitively, this calculates an activation for the central vertex, each neighbor vertex, and each edge between a neighbor and the central vertex separately.
It is the inclusion of edge activations that allows each neighbor to be distinguished from the others on the basis of its relationship to the central vertex.
This variant is called sum coupled because the neighbor vertex and edge activations are added together.
Because of this, the direct association between a neighbor its edge is lost.
A coupling which maintains the association is product coupling, which output is:
\begin{equation}
z_i = \sigma \bigg( W^{\textsc{c}} x_i + \frac{1}{|\mathcal{N}_i|}\sum_{j \in \mathcal{N}_i} (W^{\textsc{n}} x_j \odot W^{\textsc{e}} A_{ij}) + b \bigg),
\label{eq:prod_coupling}
\end{equation}
where $\odot$ denotes the elementwise product between two vectors or matrices. 
This allows a neighbor's influence on the overall activation to be modulated by its relationship to the central vertex.
For protein graphs, this means neighboring residues will contribute more or less to the overall activation, depending on their distance from and relative orientation to the central vertex, with the precise modulation determined by the edge activations.

Both sum coupled and product coupled graph convolutions are invariant to rotations or translations in space, don't impose an ordering in the neighbors or a correspondence between receptive fields of any kind, allow for different numbers of neighbors, and account for the different relationships between neighbors and the central vertex. 
The receptive fields are always defined around a central vertex, so the results of convolution can be applied to that vertex.
This retains the graph structure after each convolution, so convolutional layers are stackable, as with images.

A note on receptive fields: protein graphs are complete and embedded in a metric space, so this thesis defines a receptive field using a fixed number of closest neighbors to the central vertex.
A receptive field can also be defined using a threshold $\delta>0$ such that all vertices closer to the central vertex than the threshold are included in the receptive field.
All neighbors in a receptive field are guaranteed to share an edge with the central vertex, allowing the application of equations \ref{eq:sum_coupling} and \ref{eq:prod_coupling}.
For incomplete graphs, a receptive field can be defined as all vertices within $k$ hops of the central vertex. 
If $k=1$, both versions of graph convolution can directly be applied.
If $k>1$, then product coupled graph convolution can't be directly applied to neighbors more than 1 hop away from the center vertex, since they share no edge with the center. 
Though there are ways to adapt product coupled graph convolution in this situation, they are not the focus of this thesis.
%TODO: say more?

These graph convolutions allow the detection of local patterns on a single graph, and produce a new representation at each vertex.
Partner specific protein interaction, however, requires classifying pairs of residues in different proteins (vertices in different graphs), which is essentially making predictions on vertices in the product graph. 
Such predictions are made using a pairwise neural network architecture.

\section{Pairwise Neural Network Architecture}
A pairwise architecture is comprised of three main sections: the \textit{pre-merge} (or \textit{leg}) layers, the \textit{merge} layer, and the \textit{post-merge} layers.
See Figure \ref{fig:pairwise_arch1} for a graphical depiction.
Convolution occurs for on eprotein at a time, so this is performed in the pre-merge layers.
A key requirement for the pairwise architecture is symmetry, since the prediction for a pair of residues should be irrespective of which leg is used for each protein.
To ensure symmetry in the pre-merge layers, weights are shared between layers in the different leges.
After one or more convolutions, each graph contains a new representation at each vertex. 
The merge layer then combines the vertex representations from one graph with the vertex representations from the other into pairs. 
To maintain symmetry, This merge process should also be symmetric.
For example, the elementwise sum, elementwise product, and outer-product are all commutative and therefore produce symmetric output.
Another option is to combine pairs asymmetrically, such as concatenating the two representations together, but then average the predictions from each ordering of the representations.
%TODO: talk about pairwise kernels?
After merging, the combined representation for each pair of residues is passed through a number of post-merge layers.
The data are represented as pairs of residues at this point. 
Theoretically, graph convolutions could be performed at this stage as well, this time in the product graph.
However, the computational and memory requirements of doing so prove prohibitive since the number of convolutions and the number of neighbors in each convolution increases exponentially.
Hence the work in this thesis performs no convolution after merging.
The final layer has a single output for each pair indicating the prediction for that pair
This output is be compared to an encoded label indicating whether or not the pair constitute part of the true interface. 

\begin{figure}
	\centering
	%\begin{center}
	\includegraphics[width=0.8\textwidth]{pairwise_network2.pdf}
	%\end{center}
	\caption{.}
	\label{fig:pairwise_arch1}
\end{figure}

This chapter has presented protein graphs, novel graph convolutions, and pairwise neural network architectures, all of which are components in this thesis' approach to performing partner specific protein interface prediction.
Chapter \ref{chap:experiments} describes the experiments that were performed and gives discussion of the results.


\chapter{Experiments}
\label{chap:experiments}

The proposed pairwise graph convolutional neural networks were evaluated and compared to prior work in both partner specific interface prediction and graph convolution.

\section{Data}

Each model was trained, validated, and tested using complexes from the Docking Benchmark Dataset (DBD) version 5.0~\cite{vreven2015}, a collection of 3D crystalline structures of transient complexes in both the bound and unbound formations. 
DBD is non-redundant in the sense that no two proteins in one complex are in the same Structural Classification of Proteins (SCOP)~\cite{murzin1995} families as two proteins in another complex, respectively.
This ensures the dataset is appropriate for testing since methods that perform better on certain SCOP families than others are not unfairly rewarded or penalized.
Each protein is classified into either \emph{rigid body}, \emph{medium difficulty}, or \emph{difficult}, based on the degree of conformational change of the interfaces.
This is quantified using the root mean squared deviation of the alpha carbons after unbound and bound structures are superimposed on one another. 
Training and validation were performed using the 175 complexes that were present in version 4.0 of DBD, whereas the 55 new complexes added in DBD 5.0 were reserved for testing. 
Training and validation sets were created using a 80\%/20\% partition on the DBD v4.0-only complexes, such that the class proportions in each partition were roughly equivalent.
Hence there were 140 training complexes and 35 validation complexes.
Table \ref{tab:examples} indicates the number of positive and negative examples used from each dataset.

\begin{table}
	\centering
	\begin{tabular}{l l c r r}
		\toprule
		Phase & Data Set & Complexes & Positive examples  & Negative examples \\ 
		\midrule
		\multirow{2}{*}{Model Selection} 
			& Train      & 138       & 12,632 (9.1\%)     & 126,320 (90.9\%) \\
			& Validation & 34        & 3,042 (0.2\%) 		& 1,868,270 (99.8\%) \\
		\midrule
		\multirow{2}{*}{Testing}
			& Train + Validation & 175 & 16,004 (9.1\%) & 160,040 (90.9\%) \\
			& Test       & 55        & 4,871 (0.1\%)      & 4,953,446 (99.9\%) \\ 
		\bottomrule
	\end{tabular}
	\caption{Number of complexes and examples (residue pairs) for each phase of experimentation. Negative examples were down sampled to a ratio of 10:1 with positive examples when training, but testing included all examples. During model selection, three complexes were excluded from the training and validation sets due to a software bug and missing features. These complexes were eventually included for the testing phase. \label{tab:dataset_size}}
	\label{tab:examples}
\end{table}

3D structures are contained in Protein Data Bank (PDB) files, which contain atomic coordinates for each amino acid in the complexes.
True interfaces were determined using the bound formations, where two amino acids are assumed to be interacting if they are within 6 \AA{} of each other, consistent with prior work~\cite{ofran2007, ahmad2011, minhas2014}.
This permits a residue to interact with multiple partners in the interface.

\subsection{Vertex and Edge Features}
Prediction must be made using the unbound formation of each protein, since the bound formation is not known a priori.
Hence unbound PDB files were used to generate a graph representation of each protein, with vertices indicating residues and edges indicating the relationships between residues.
Features were then computed for each vertex and edge using sequence and structure information from the protein.
Vertex features pertain to the degree of residue conservation, accessible surface area, depth within the protein, and protrusion.
Edge features capture the distance between two residues and their relative orientation as well.
More detail on each feature can be found in Appendix \ref{appendix:features}.

\subsection{Missing Data}
In some cases, features could not be calculated for a residue, resulting in missing values.
For 13 residues, the atomic coordinates of the central alpha carbon were not present in the PDB files, so the Half Sphere Amino Acid Composition, Residue Depth, and $\mathrm{C C_{\alpha} O}$ angle could not be calculated.
Most features rely on third party software to calculate the features from the raw sequence and structure information.
For the proteins with DBD codes 2B42 and 3R9A, features related to the protrusion index could not be calculated for the receptor due to a software fault.
These complexes were omitted during model selection but re-introduced before testing.
Two other complexes, 1NW9 and 1PPE, were also excluded during model selection due to a software bug.
This bug was eventually fixed, so these were also included during testing. 
During model selection, missing values were imputed using the feature mean within a proteins, whereas for testing, the global median (within the training, validation, or test set) was used.
In total there were 138 training and 34 validation complexes during model selection, and there were 175 training and 55 test complexes during testing.
%TODO: find total number of nans?

\section{Experiments}

Experimentation was conducted in two phases: model selection, and testing.
The model selection phase concerned the selection of model \emph{hyper-parameters}, those details which are not explicit model parameters but nevertheless affect the behavior of the network.
For this phase, pairwise graph convolutional networks were repeatedly trained using the training set and evaluated on the validation set.
Once model selection was complete, the final set of model hyper-parameters was used to train and test the most promising hyperparameter configurations and compare them to existing methods.
For this phase, the model was trained on the combined training and validation sets and tested on the test set.


\subsection{Model Selection}

Not only do a network's parameters ($\Theta$) help determine the network's output, but the set of hyper-parameters as well.
Hyper-parameters comprise all model specifications besides the parameters themselves, such as the number and types of layers, number of units/filters in each layers, choice of nonlinear activation function, loss function, and training details.
Hyper-parameters are typically chosen either by some automated process, or by manual exploration.

Automated hyper-parameter selection processes include a randomized search, grid search, or a sequential, model based optimization (SMBO) algorithm such as a Gaussian Process or Tree Structured Parzen Estimator~\cite{bergstra2011}.
The advantage of these approaches is their ability to automatically search a hyper-parameter space without human intervention. 
SMBO approaches are able to estimate the performance "response surface" in hyper-parameter space in order to efficiently find the optimum value.
The disadvantage of these approaches is they sometimes require a large number of iterations in order to converge to optimal parameters, and it may be difficult to incorporate human prior knowledge about the optimal set of hyper-parameters.
Some experiments were performed with SMBO algorithms, but none produced hyper-parameters which outperformed results found by human exploration.
One hindering factor for the automated approaches was the sheer number of parameters being explored, making it difficult to sample the space in an efficient way.
%TODO: report number of hyperparameters tried?
Therefore automated approaches were abandoned in favor of a manual search.

Several hyper-parameters were explored manually by training/validating networks under different settings of the hyper-parameters.
This exploration occurred in a one-hyper-parameter-at-a-time fashion, alleviating the exponential problems encountered with the automated algorithms, but also limiting the ability to explore the interactions between different hyper-parameters.
Following is a list of different hyper-parameters which were explored and the general trends encountered in each.
\hl{TODO: fill in details in this list}
\begin{itemize}
	\item Number of convolutional layers:
	\item Number of convolution filters:
	\item Receptive field size:
	\item Number of dense layers after merge:
	\item Parameter inizialization: weights and biases
	\item Nonlinear activation function:
	\item 
	\item Loss function:
	\item Update algorithm:
\end{itemize}

The cumulative findings of and intuition gained from the manual model selection procedure informed the final set of experiments which were run in the testing phase of experimentation.

\subsection{Testing}

Final model testing was performed for both sum coupled and product coupled graph convolutions.
For each variant, receptive field sizes of 11 and 21 were tested (the central vertex plus 10 or 20 neighbors, respectively), with one, two three, and four graph convolutional layers. 
In total, 16 hyper-parameter conditions were tested, all of which used the following training scheme:
The number of filters for networks with one convolutional layer was 256, Likewise for two, three, and four convolutional layers the number of filters was (256, 512), (256, 256, 512), and (256, 256, 512, 512) respectively. 
After merging examples, dense layers of 256 and 1 units respectively were applied, with the latter producing the model output.
Loss was computed using weighted cross entropy, a measure of the dissimilarity between two probability distributions. 
Since this is a binary classification problem, the cross entropy for a residue pair $x_i$ is defined as: 

\begin{equation}
\mathbb{H}(x_i, y_i | \Theta) = - r ~ y_{i0} ~ log\big(f_0(x_i|\Theta)\big) - y_{i1} ~ log\big(f_1(x_i|\Theta)\big)
\label{eq:weighted_ce}
\end{equation}

\noindent
where $f = (f_0, f_1)$ is the softmax network output, interpreted as class probabilities, $y_i = (y_{i0}, y_{i1})$ is the one-hot class label, and $r$ is the ratio of positive to negative examples, which ensures that the set of positive examples and the set of negative examples contribute equally to the loss function.
The overall loss function is then:

\begin{equation}
\mathbb{L}(\Theta | \{x_i\}, \{y_i\}) = \sum_{i} \mathbb{H}(x_i, y_i | \Theta),
\end{equation}

\noindent
which sums the cross entropy from each residue pair prediction.


All bias weights were initialized to zero.
All non-bias weights were initialized by drawing from a uniform distribution between $-\tau_0$ and $\tau_0$, where $\tau_0=\frac{1}{\sqrt{n_{in}}}$ and $n_{in}$ is the number of inputs to the layer.
This is similar to the initialization proposed by He, et at (2015)~\cite{he2015}, except they differ by the small multiplicative constant, $\sqrt{6}$.
Stochastic gradient descent was performed with a learning rate of 0.1, mini-batch size of 128 pairs, for 80 epochs. 
Dropout with probability $p=0.5$ was performed during training.

The networks were implemented in TensorFlow~\cite{abadi2015} v1.0.1 and trained using a single NVIDIA GTX 980 or GTX TITAN X GPU.
Training time varied from 9 to 46 minutes, depending on network size, GPU card, and computer resource availability.
Figure \ref{fig:train_times} shows that training time increases roughly linearly in the number of convolutional layers, and also that larger receptive fields train more slowly.
This is consistent with the fact that the no convolution case essentially has a receptive field of size zero.

\begin{figure}
	\includegraphics[width=0.8\textwidth]{training_times.png}
	\caption{Training time for pairwise graph convolution neural network, during testing. Times are shown for sum and product coupled convolutions, receptive fields of size 11 and 21, and 1-4 convolutional layers. Training time scales roughly linearly with the number of layers.
	\label{fig:train_times}}
\end{figure}

Graph convolutional networks were compared to PAIRpred, an existing state-of-the-art partner specific interface prediction algorithm based on an SVM with custom symmetric pairwise kernels, proposed by Minhas et al, (2014)~\cite{minhas2014}.
The original PAIRpred publication used different complexes, slightly different features, and a different experimental protocol, so it was run with the same features and data that were used in this work for the purpose of comparison.
Grid search with five-fold cross validation was used to select the optimal kernel and soft margin parameters.

The proposed graph convolution layers were also compared against diffusion convolutions, a spatial graph convolution proposed by Atwood and Towsley, (2016), designed to simulate a diffusion process across each vertex of an input graph~\cite{atwood2016}.
Diffusion convolutions generate a power series of a weighted adjacency matrix, where each matrix of the power series is the adjacency matrix raised to a unique power.
In their convolutions, all powers up to a certain limit (the maximum number of "hops").
This approach only supports a single weight on each edge, so in this case Gaussian normalized distances were used and relative angle was omitted.
Manual model selection was used to identify the best values of the standard deviation parameter in the Gaussian normalization.
These convolutions do not downsample the input, so it is possible to stack multiple convolutions on top of one another.
However, additional convolutional layers did not improve performance, so only results from a single layer are shown. 

\section{Metrics}

Biologists can often learn a lot of useful information from just a few pairs that are known to be a part of the interface. \hl{cite?}
In the partner specific problem, even this task is difficult, since the number of negative pairs far outweighs the number of positive pairs.
An imperfect classifier may succeed in giving high scores to interface pairs and some non-interface pairs as well.
The more assured a researcher can be that top scoring pairs are from the interface, the more useful is the classifier.

While measures like accuracy, precision, recall, specificity, and F-score are common for many classification problems, they are dependent on the choice of a particular threshold.
A metric more appropriate for interface prediction would be the area under the receiver operating characteristic curve (AUC), because it does not rely on a particular choice of threshold.
Instead, it summarizes the true positive rate across all false positive rates, indicating threshold independent performance of the classifier.

In addition, such an extreme class imbalance can imply that AUC is misleading, since the movement of a single positive example can heavily influence the true positive rate compared to the effect that a single negative example has on the false positive rate.
An alternative measure which more directly captures the usefulness of classifiers is the rank of the first positive prediction (RFPP), where the rank is computed after ordering pairs by classifier score.
Ideally, the top ranking prediction is truly part of the interface, in which case RFPP is 1.

Interface prediction occurs at the protein complex level, so AUC and RFPP should be calculated with respect to each complex. 
The performance of a classifier on a set of complexes can then be summarized using median AUC and RFPP across the complexes.
These metrics closely mimic those used in the PAIRpred paper~\cite{minhas2014}, except that AUC is being calculated individually for each complex instead of overall.


\chapter{Results}
\label{chap:results}





\section{Results}

This section first summarizes the effect of protein size on network training time.
Then is expores the behavior of sum and product coupled graph convolutions.
Finally, it compares to PAIRpred and the other variants of graph convolution.


\subsection{Training Time}

Figure \ref{fig:train_times} shows a log-log plot of training time as a function of the number of residue pairs in a given complex.
The lines are roughly linear in this space, indicating a power-law relationship.
Fitting lines to the data shows that the power is slightly over 1.0 in all cases, showing that training time increases roughly linearly in the number of pairs (see also Table \ref{tab:training_time_powers}), as expected.
This is due solely to the pairwise nature of the problem, and in non-pairwise cases, the training time would likely scale linearly in the number of vertices.



that training time increases roughly linearly in the number of convolutional layers, and also that larger receptive fields train more slowly.
This is consistent with the fact that the no convolution case essentially has a receptive field of size zero.

\begin{figure}
	\includegraphics[width=0.8\textwidth]{training_times.png}
	\caption{Training time for pairwise graph convolution neural network, during testing. Times are shown for sum and product coupled convolutions, receptive fields of size 11 and 21, and 1-4 convolutional layers. Training time scales roughly linearly with the number of layers.
		\label{fig:train_times}}
		\end{figure}

\subsection{Sum and Product Coupling Performance}

\begin{table}
	\begin{center}
		\begin{tabular}{lccccc}
			\toprule
			\multirow{2}{*}{Method} &
			Receptive Field & \multicolumn{4}{c}{Layers Before Merge} \\
			& Size & 1 & {2} & {3} & {4} \\
			\midrule
			No Convolution & N/A & \textbf{0.815} & 0.812 & 0.800 & 0.811  \\\cline{1-6}
			\multirow{2}{*}{Sum Coupled} & 11 & 0.868 & 0.889 & 0.882 & 0.884 \\
			& 21 & 0.875 & \textbf{0.903} & 0.880 & 0.890 \\\cline{1-6}
			\multirow{2}{*}{Product Coupled} & 11 & 0.856 & 0.869 & 0.885 & 0.868 \\
			& 21 & 0.863 & 0.876 & 0.896 & \textbf{0.899} \\
			\bottomrule
		\end{tabular}
		\caption{Median area under the receiver operating characteristic curve (AUC) across all complexes in the test set for two variants of graph convolution, \textit{sum-coupled} and \textit{product-coupled}. Results are shown for two different sizes of receptive field, 11 and 21, for different numbers of convolutional layers before the pairwise merge operation. Bold faced values indicate best performance for each method.}
		\label{tab:med_auc}
	\end{center}
\end{table}

Table \ref{tab:med_auc} shows the results of experiments involving sum coupled and product coupled graph convolution, as well as no graph convolution.
Comparing convolution to no convolution reveals that convolution is beneficial.
Specifically, incorporating information from neighboring residues helps indicate whether a particular residue is part of an interface, which is consistent with the biological properties of interfaces.
It's also clear that a receptive field of size 21 is generally better than 11.
For larger receptive fields, this trend does not continue, as performance drops (data not shown).
Interestingly, this value is approximately the size of a typical interface \hl{get number from basir?}.
When using convolution, performance improves with network depth up to a point, then decreases (this is true for product coupling as well, though the data are not shown).
In contrast, networks without convolution are best with only one pre-merge layer.
This suggests that depth alone does not improve performance, but when convolution is performed, a useful hierarchical representation is learned.
Other applications of deep learning have seen this same trend of increasing and decreasing performance.
Possible explanations include insufficient training data, a vanishing gradient, and a problem which is difficult to optimize~\cite{he2015}.
This results also indicate that in most cases, sum coupled graph convolution outperforms product coupled convolution.
This observation is counterintuitive under the premise that product coupling better exploits structure in the input by associating neighbor and edge through elementwise product.
However, AUC is just one measure of performance, and so RFPP must also be examined.

\begin{table}
	\begin{center}
		\begin{tabular}{lccccc}
			\toprule
			\multirow{2}{*}{Method} &
			Receptive Field & \multicolumn{4}{c}{Layers Before Merge} \\
			& Size & 1 & {2} & {3} & {4} \\
			\midrule
			No Convolution & N/A & \textbf{48} & 55 & 53 & 66 \\\cline{1-6}
			\multirow{2}{*}{Sum Coupled} & 11 & 32 & 28 & 70 & 86 \\
			& 21 & \textbf{26} & 37 & 56 & 63 \\\cline{1-6}
			\multirow{2}{*}{Product Coupled} & 11 & 30 & 46 & 26 & 51 \\
			& 21 & 26 & \textbf{25} & 36 & 37 \\
			\bottomrule
		\end{tabular}
		\caption{Median rank of the first positive prediction (RFPP) across all complexes in the test set for two variants of Graph Convolutional Networks (GCN), \textit{sum-coupled} (SC) and \textit{product-coupled} (PC). Results are shown for two different sizes of receptive field, 11 and 21, for different numbers of convolutional layers before the pairwise merge operation. Bold faced values indicate best performance for each method (lower is better). For a given receptive field size and number of layers, product coupling performs better than sum coupling except for one case.}
		\label{tab:med_rfpp}
	\end{center}
\end{table}

Table \ref{tab:med_rfpp} parallels Table \ref{tab:med_auc} but shows RFPPs instead of AUCs.
As before, convolution generally outperforms no convolution, the few exceptions being sum coupled variants with a smaller receptive field.
In this case, however, product coupling outperforms sum coupling for all but one case.
If we accept that product coupling \emph{does} detect more specific patterns than sum coupling, then a lower RFPP may indicate that this property allows the network to better detect the specific patterns which occur in the most  obvious interface pairs.
Unfortunately, the best RFPP performance does not coincide with the best AUC performance, except when not convolving.
This is unsurprising, since the cross-entropy loss function leads to optimization of performance on \emph{all} pairs, not just the top scoring ones.
In other words, RFPP is not being explicitly optimized in this example, whereas AUC is more closely related to the quantity being optimized.

\begin{figure}
	\includegraphics[width=\textwidth]{med_auc.png}
	\caption{Median area under the receiver operating characteristic curve (AUC) across all complexes in the test set, separated by complex class. Sum and product coupling are shown for two receptive field sizes each (11 and 21), as well as no convolution, for 1-4 pre-merge layers. Product coupling performs better for difficult complexes, but worse overall because ther are far more rigid and medium difficulty complexes.
		\label{fig:med_auc}}
\end{figure}

To understand the behavior of each method in more detail, we can separate performance by the difficulty class of the test proteins.
Each chart in Figure \ref{fig:med_auc} shows performance for a single difficulty class, including rigid, medium difficulty, and difficult, with 33, 16, and 6 complexes respectively.
Here it appears that sum and product coupling are closely matched for rigid and medium difficulty, with sum coupling slightly outperforming product coupling in most cases.
A much more significant difference is seen for the difficult complexes, where product coupling is clearly doing better, particularly for 2 and 3 layers.
This implies that the better performance of sum coupling overall is driven by the distribution of difficulty of complexes in the test set.
In the presence of more difficult complexes, it is likely that product coupling would emerge as the clear winner, restoring our original intuition.

\begin{figure}
	\includegraphics[width=0.8\textwidth]{sum_20_2_histo1.png}
	\caption{Histogram of area under the receiver operating characteristic curve (AUC) for complexes in the test set, colored by difficulty class. Scores are from sum coupled graph convolution with two layers and receptive field size 21, which had the highest median AUC of all methods.}
	\label{fig:histo1}
\end{figure}

For another picture of performance across difficulty classes, we can look at histogram of AUCs, as shown in Figure \ref{fig:histo1}.
These AUCs are heavily skewed left, justifying the choice of median for a summary measure.
Surprisingly, there is no clear divide between rigid, medium difficulty, and difficult classes.
In fact, the worst AUC is a rigid complex, and at least one difficult complex achieves AUC above 0.9.
It appears that the distinguishing characteristic between classes is the number of complexes that achieve above 0.95 AUC.
These "trivial" complexes are most frequent in the rigid class, less so in the medium difficulty class, and absent for the difficult class.
%TODO: so what is it about complexes that makes them easy or hard? look at interface size?	

\begin{figure}
	\includegraphics[width=\textwidth]{med_rfpp.png}
	\caption{Median rank of the first positive prediction (RFPP) across all complexes in the test set, separated by complex class. Vertical axes are log scaled. Sum and product coupling are shown for two receptive field sizes each (11 and 21), as well as no convolution, for 1-4 pre-merge layers. Lower RFPP is better. Best performance on rigid complexes is achieved with just one layer for all networks. As difficulty increases, so does the number of layers needed to achieve best results.
		\label{fig:med_rfpp}}
\end{figure}

RFPP can also be separated by difficulty class.
From figure \ref{fig:med_rfpp} we can observe some heterogeneus behavior across methods, but there are some trends worth noting.
For each class, there appears to be a favored number of layers where a notable dip (improvement) in RFPP is observed. 
This optimal depth appears to increase with complex difficulty, suggesting that harder complexes require more layers to achieve best performance.
This suggests that an ensemble approach with varying depths could perform well on complexes of any difficulty. 
In most cases, product coupling performs the best, except for rigid bodies and difficult complexes with few layers.



\subsection{Comparison to Other Methods}

\begin{table}
	\begin{center}
		\begin{tabular}{l c c c c }
			\toprule
			
			\multirow{2}{*}{Method} & \multicolumn{4}{c}{Median AUC} \\
			& Overall & Rigid & Medium & Difficult \\
			\midrule
			PAIRpred      & 0.863        & & & \\
			\midrule
			2-hop DCNN ($\sigma=2$\AA{}) & 0.782 & 0.775 & 0.821 & 0.753 \\
			2-hop DCNN ($\sigma=4$\AA{}) & 0.801 & 0.820 & 0.817 & 0.681 \\
			\midrule
			5-hop DCNN ($\sigma=2$\AA{}) & 0.838 & 0.849 & 0.847 & 0.793 \\ 
			5-hop DCNN ($\sigma=4$\AA{}) & 0.819 & 0.832 & 0.867 & 0.740 \\ 
			\midrule
			
			Sum Coupled (2 layers) & 0.903 & 0.903 & 0.941 & 0.800 \\ 
			\bottomrule
			\\
		\end{tabular}
		\caption{Comparison with existing classification methods. 
			PAIRpred is a state-of-the-art interface prediction method~\cite{minhas2014}, and DCNN is the diffusion convolution method~\cite{atwood2016}.
			For DCNN, there was a single convolutional layer before merging, as more layers did not improve the performance (data not shown). DCNN performed better when the adjacency matrix consisted of a Gaussian function of distance, with a standard deviation of 2\AA{} and 4\AA{}, for 2 and 5 hops.}
		\label{tab:results_compare}
	\end{center}
	%\end{minipage}
\end{table}

Figure \ref{tab:results_compare} compares the median AUC of PAIRpred, diffusion convolution (DCNN) with 2 and 5 hops, as well as for distance Gaussian standard deviations of 2 and 4 \AA{}, and the best sum coupled results.
The proposed graph convolutions outperform both existing methods, \hl{in every class (check for PAIRpred when I get numbers, also write more about PAIRpred)}.
For DCNN, 5 hops performs better than 2 hops, presumably because it allows information to propagate further across the graph.
The Gaussian standard deviation directly determines the weights on the edges between vertices, therefore the strength of diffusion along that edge.
In this context, smaller values limit diffusion to a localized neighborhood for each hop, whereas larger values allow diffusion across longer distances.
In the 2 hop DCNN, The larger standard deviation allows more diffusion to occur across the graph, compensating for the limited number of hops.
This explains the overall better performance for $\sigma=4\AA{}$, although for medium and difficult complexes, $\sigma=2\AA{}$ appears to do better. 
In these cases, the degree of conformational change is greater, making information at larger distances less reliable for prediction for a residue.
In contrast, the 5 hop DCNN performed better for $\sigma=2AA{}$, suggesting that the larger number of hops allows sufficient information propagation, eliminating the need for diffusion across greater distances for each hop.
This trend is broken for the medium complexes, however.

These results indicate that sum coupled and product coupled convolutions are learning useful representations of each residue that are useful in performing interface prediction.
Their performance is superior to a state-of-the-art interface prediction method and an existing graph convolution approach. 



\subsection{Filter Visualization}

\hl{develop better intuition of behavior of models.}

\chapter{Future Work}
\label{chap:future}

The work in this thesis is pertinent to emerging methods in machine learning as well as current problems in structural bioinformatics.
As such, extensions of this work can be made in several distict veins, some focused on model formulation and training procedure, others on other possible applications.
Several of these extensions are documented below, categorized as either extensions of method or extensions of application.

\section{Extensions of Method}

\subsection{Double Coupling and Ensemble Approaches}

As discussed in Chapter \ref{chap:experiments}, sum coupling outperforms product coupling on rigid and medium difficulty complexes, whereas product coupling is better for difficult complexes.
Therefore it would be reasonable to incorporate both sum and product coupling into a single convolution operation:

\begin{equation}
h_i(x | W^\textsc{c}, W^\textsc{n}, W^\textsc{e}, b) = \sigma \bigg( W^{\textsc{c}} x_i + \frac{1}{|\mathcal{N}_i|}\sum_{j \in \mathcal{N}_i} (W^{\textsc{n}} x_j) \odot (W^{\textsc{e}} A_{ij}) + W^{\textsc{n}} x_j + W^{\textsc{e}} A_{ij} + b \bigg),
\label{eq:double_coupling}
\end{equation}

\noindent
where here the same weight matrices are used in both the sum coupling and product coupling components to help prevent overfitting. 
Alternately, separate weight matrices could be used to increase the model's expressive power.

Recall that concerning the RFPP metric, the optimal network depth increased with complex difficulty, where more layers were needed for complexes of greater difficulty. 
This suggests that a ensemble model with multiple networks of varying depth could perform  better than each individual network.
Alternative ensemble approaches like \emph{boosting} or \emph{bagging} may benefit as well.
The former approach trains a sequence of "weak" models, with each added model being trained by giving more importance (loss weight) to examples misclassified by all existing models.
For interface prediction, this weighting can be performed on a per-complex basis, using RFPP or AUC as an indication of performance on each complex, or on a per-residue-pair basis, using the cross entropy of that specific training example.
The final prediction is a weighted average of the weak models' predictions, with the weights determined by the validation error of each respective model (with higher weight given to models with lower error).
The latter approach creates multiple datasets by sampling with replacement from the original data set, and training a different model on each sampled dataset.
This sampling can be performed at the complex level or the residue pair level.
Again, the final prediction is a weighted average of the weak models' predictions, weighted according to validation error.



\subsection{RFPP Optimization}

In practice, it may be more important to a biologist that a classifier give a single good prediction of the interface location, rather than predict the entire interface with a high degree of accuracy.
This was the original motivation for RFPP as a performance metric.
However, the existing model uses a loss function which incorporates every training examples, not just the top positive prediction, hence the model is optimized for all examples.
It could be that optimizing the model for all examples sacrifices performance for the top predictions, and that directly optimizing RFPP could yield better results with respect to RFPP.
A modified version of the loss function is:

\begin{equation}
\mathbb{L}(\Theta | {x_i}, {y_i}) = - \max_{i} \big(y_{i1} f_1(x_i)\big),
\label{eq:rfpp_optimize}
\end{equation}

\noindent
which includes only the score of the maximum performing positive example.
Hence when the loss is minimized, the performance of the highest performing positive example will be maximized.
This loss function is not differentiable, but sub-gradient methods and differentiable alternatives exist which circumvent this fact.
This method is currently being investigated by another graduate student in Asa Ben-Hur's research group.

\section{Additional Data Sources}

The success of deep learning methods has been attributed to large volumes of data and deep architectures~\cite{krizhevsky2012}.
For interface prediction, despite the large volume of recorded protein structures, precious few complexes have been annotated in bound and unbound forms for use in model training and testing. 
This dependence on small curated subsets of available proteins has potentially limited the full leveraging of deep learning methods, however some opportunities exist for enlarging the training and testing data.

The Docking Benchmark Dataset was conceived as a method to evaluate docking methods, and correspondingly was carefully constructed to be non-redundant with respect to SCOP families, in order to give a fair evaluation. 
However for training purposes, it may be useful to include redundant proteins simply to provide the model with more training data. 

The Critical Assessment of PRediction of Interactions (CAPRI) is an annual competition aimed at evaluating protein-protein docking methods~\cite{janin2003}, and could also be used to help train or evaluate partner-specific protein interface prediction methods. 

\section{Unsupervised Pretraining}

Another approach to the problem of limited annotated complexes is to utilize the vast quantity (>125,000) of recorded protein structures via unsupervised pre-training.
Hinton and Salakhutdinov (2006)~\cite{hinton2006b} proposed a greedy layer-wise training algorithm for a particular form of neural network called an \emph{autoencoder}.
Training an autoencoder consists of training a sequence of Restricted Boltzmann Machines (RBMs) using unlabeled training data, then "unfolding" these RBMs into encoding and decoding layers which are meant to reconstruct the original input, after creating a compressed representation. 
These weights can then be fine tuned using back propagation, where the output is further trained to reconstruct the input.
The encoding portion of the autoencoder can then be used to create efficient representations of the input that can help in supervised tasks like classification~\cite{hinton2006b, bengio2007}.

Masci et al (2011)~\cite{masci2011} proposed convolutional autoencoders (CAEs) which perform convolution to encode an image and \emph{deconvolution} to decode the image.
Deconvolution is simply a convolution operation performed on the result of an encoding convolution, with weights being tied between convolution and deconvolution layers.
For $k$ filters of size $m \times m$ and $c$ channels, the corresponding deconvolution would consist of $c$ filters of size $m \times m$ and $k$ channels, where the weights have been reflected in both spatial dimensions (e.g. the weight(s) for the lower left pixel of the receptive field is now applied to the upper right pixel).
The spatial reflection allows a particular weight(s) to carry a specific meaning, namely it characterizes the relationship between a particular pixel in an image and a pixel in its encoding.
To illustrate this, Figure \ref{fig:deconv} shows a deconvolution for a single filter and single channel.
CAEs are trained in the same greedy, layerwise fashion as conventional autoencoders.

\begin{figure}
%	\includegraphics[width=0.8\textwidth]{deconv_example.png}
	\caption{
		\label{fig:deconv}}
\end{figure}


The concept of deconvolution can be applied to graph convolutions as well.
Note that when reflecting image convolution filters, the center weights in the same position.
In the same way, graph deconvolutions retain the same weight matrices for the central vertex. 
Furthermore, since all neighbors use the same weights, there is no spatial reflection to be performed.
Deconvolution simply consists of transposing the center and neighbor weight matrices so that channels become filters and filters become channels.
For sum coupling, deconvolution becomes:

\begin{equation}
\hat{x_i}(h_i | \tilde{W}^\textsc{c}, \tilde{W}^\textsc{n}, W^\textsc{e}, b) = \sigma \bigg( \tilde{W}^{\textsc{c}} h_i + \frac{1}{|\mathcal{N}_i|}\sum_{j \in \mathcal{N}_i} (\tilde{W}^{\textsc{n}} h_j + W^{\textsc{e}*} A_{ij}) + b^* \bigg),
\label{eq:sum_deconv}
\end{equation}

\noindent
where $\hat{x}$ denotes the reconstruction of $x$ after deconvolving, $h_i$ is the convolved representation at vertex $i$, $\tilde{W}^\textsc{c}$ is the transpose of $W^\textsc{c}$, likewise for $\tilde{W}^\textsc{n}$ and $W^\textsc{n}$, and both $W^{\textsc{e}*}$ and $b^*$ are weight matrices with unshared weights. 
Note that graph convolutions generate representations on vertices of the graph, not the edges, so the non-encoded edge information must be used, and the weight matrix $W^{\textsc{e}*}$ cannot be tied to the encoding matrix $W^{\textsc{e}}$.

Alternately, edge representations could be created in a separate convolution operation, where an edge's receptive field is simply the incident vertices:

\begin{equation}
h_{ij}(A_{ij} |W^{\textsc{ee}}, W^{\textsc{v}}, b_e) = \sigma\bigg( W^{\textsc{ee}} A_{ij} + \frac{1}{2}\big(
W^{\textsc{v}} x_i + W^{\textsc{v}} x_j \big) + b_\textsc{e} \bigg),
\label{eq:edge_conv}
\end{equation}

\noindent
where $h_{ij}$ is the representation of edge $(i, j)$ , $W^{\textsc{ee}}$ is the weight matrix associated with the edge, and $W^{\textsc{v}}$ is the weight matrix associated with the incident vertices.
In this case, vertex convolution can use the encoded edge representations and associated weights.
Equation (\ref{eq:sum_deconv}) then becomes:

\begin{equation}
\hat{x_i}(h_i | \tilde{W}^\textsc{c}, \tilde{W}^\textsc{n}, \tilde{W}^\textsc{v}, b) = \sigma \bigg( \tilde{W}^{\textsc{c}} h_i + \frac{1}{|\mathcal{N}_i|}\sum_{j \in \mathcal{N}_i} (\tilde{W}^{\textsc{n}} h_j + \tilde{W}^{\textsc{V}} h_{ij}) + b^* \bigg),
\label{eq:sum_deconv2}
\end{equation}

\noindent
where $\tilde{W}^{\textsc{V}}$ is the transpose of $W^{\textsc{V}}$.
Edges can also be deconvolved using the edge and vertex representations:

\begin{equation}
\hat{A_{ij}}(h_{ij} |\tilde{W}^{\textsc{ee}}, \tilde{W}^{\textsc{e}}, b) = \sigma \bigg( \tilde{W}^{\textsc{ee}} A_{ij} + \frac{1}{2}\big(
\tilde{W}^{\textsc{e}} h_i + \tilde{W}^{\textsc{e}} h_j \big) + b^{*}_\textsc{e} \bigg).
\label{eq:edge_conv}
\end{equation}

\noindent
This added weight sharing and symmetry between convolution and deconvolution operations may allow training of deeper networks which recognize more sophisticated structures.




\section{Model}

\subsection{Simplicial Complex Convolution}

- concept of edge coupling was to differentiate different neighbors, so not all the same. 
	this accounts for relationship of neighbors to center vertex. 
- In image convolutions, groups of "neighbor" pixels are close to one another, so that collectively they may indicate the presence of something interesting in a part of the receptive field. 
	this is referring to the relationships of neighbors to one another. 

	To capture this we can create a simplicial complex based on some threshold, average activations over vertices for each simplex in the neighborhood, and then max over all simplices to capture interesting "regions" of the receptive field. 
		- generalization of graph convolution.


\section{Other Problems}

scoring of putative dockinng solutions (graph level classification)
scoring of putative protein folding solutions (graph level classification)

QSAR
inference on Knowledge graphs







\backmatter % starts unnumbered supplementary material
%%%%%%%%%%%%%%%%%%%%%%%%%%%%%%%%%%%%%%%%%%%%%%%%%%%%%%%%%%%%%%%%

% Bibliography
%%%%%%%%%%%%%%%%%%%%%%%%%%%%%%%%%%%%%%%%%%%%%%%%%%%%%%%%%%%%%%%%

% unsorted BibTeX style
% check here for more:  https://www.sharelatex.com/learn/Bibtex_bibliography_styles
\bibliographystyle{unsrt}
\bibliography{biblio} % change readme to the name of your .tex file

\appendix % starts the appendices
%%%%%%%%%%%%%%%%%%%%%%%%%%%%%%%%%%%%%%%%%%%%%%%%%%%%%%%%%%%%%%%%

\chapter{Features}
\label{appendix:features}

The raw data associated with protein structures are found in PDB files, which contain the protein sequence of amino acids and the atomic coordinates of each atom in each amino acid.
When performing interface prediction, a number of derived features have been found useful, which are either calculated directly from the PDB data, or from other sources such as sequence databases.
Here, the term \emph{features} describes a vector of values associated with either a vertex or edge in a graph, with vertices corresponding to amino acid residues, and edges the relationships between two residues.
The term \emph{feature} then is just a single element of the vector.
Some methods described below generate more than one feature, hence they are grouped into appropriate \emph{categories}.
This appendix describes each feature category in detail, including the number of features generated, external tools, and a description of the feature.
In addition to the tools specified for each feature category, extensive use was made of Biopython~\cite{cock2009} to parse PDB files into convenient data structures. 

\section{Vertex Features}

\subsection{Windowed Position Specific Scoring Matrix (PSSM)}
\noindent
\emph{Number of features}: 20

\noindent
\emph{External tools}: PSI-BLAST~\cite{altschul1997}

\noindent
\emph{Description}:
The PSSM is a set of features constructed from protein sequence alone, without any structural information.
It captures the relative abundance of each type of amino acid in proteins which share a similar sequence in a window around the amino acid of interest.
Proteins with similar sequences to the search window are identified through PSI-BLAST, a position specific, iterative version of BLAST, the Basic Local Alignment and Search Tool~\cite{altschul1990}.
The similar sequences are then used to generate a summary in a window around the residue of interest.
For each window position and each residue type, a score is computed by comparing the observed counts of that residue type in that window position to the expected count generated by a null model.
The score is then $\text{log}(Q_i/P_i)$ where $Q_i$ is the fraction of sequences with that residue type in that position, and $P_i$ is the expected fraction based on data-dependent pseudocounts (more detail can be found in Altschul, et al (1990)).
For this thesis, only the scores at the residue of interest were considered (a window size of one), so there are 20 features, one for each amino acid type. 

\subsection{Relative Accessible Surface Area (rASA)}
\noindent
\emph{Number of features}: 1

\noindent
\emph{External tools}: STRIDE~\cite{heinig2004}

\noindent
\emph{Description}:
Relative Accessible Surface Area, or \emph{Solvent Accessible Area}, reflects the fraction of a residue that is exposed to a potential solvent.
This is computed by sliding a spherical probe of radius of 1.4\AA{} (approximating the radius of a water molecule~\cite{eisenhaber1995}) over the van der Waals surface of the protein near the residue of interest.
The Area generated by the center of the probe as it is in contact with the residue is taken to be the accessible surface area.
This is divided by the maximum possible accessible surface area to achieve a relative measure.


\subsection{Residue Depth}
\noindent
\emph{Number of features}: 2

\noindent
\emph{External tools}: MSMS~\cite{sanner1996}

\noindent
\emph{Description}:
Residue Depth is the minimum distance from the residue to the surface of the protein, as calculated by MSMS.
The distance is calculated is two ways: by taking the average distance of all non-hydrogen atoms in the residue to the surface, and by taking the distance of the alpha carbon to the surface.


\subsection{Protrusion Index and Hydrophobicity}
\noindent
\emph{Number of features}: 6

\noindent
\emph{External tools}: PSAIA~\cite{mihel2008}

\noindent
\emph{Description}:
The protrusion index indicates the degree to which a sphere (radius 10\AA{}) centered at a non-hydrogen atom is not filled with other atoms.
$V_{int}$ is the volume occupied by atoms, which is calculated as the number of atoms times the mean atomic volume found in proteins.
$V_{ext}$ is the difference between the total sphere volume and $V_{int}$. 
The protrusion index is defined as $V_{ext}/V_{int}$.
This value is calculated for each non-hydrogen atom in the residue, then the following summary statistics are taken:
\begin{itemize}
	\item \emph{Total mean}: the average over all non-hydrogen atoms in the residue
	\item \emph{Total standard deviation}: the standard deviation over all non-hydrogen atoms in the residue
	\item \emph{Side chain mean}: the average over just non-hydrogen atoms in the residue side chain
	\item \emph{Side chain standard deviation}: the standard deviation over just non-hydrogen atoms in the residue side chain
	\item \emph{maximum}: the maximum over all non-hydrogen atoms
	\item \emph{minimum}: the minimum over all non-hydrogen atoms
\end{itemize}
All of these features are normalized to values between 0 and 1 on a per-protein basis.


\subsection{Hydrophobicity}
\noindent
\emph{Number of features}: 1

\noindent
\emph{External tools}: PSAIA~\cite{mihel2008}

\noindent
\emph{Description}:
Hydrophobicity indicates the tendency for an amino acid to avoid water.
PSAIA calculates this value base on the scale defined by Kyte and Doolittle, (1982)~\cite{kyte1982}.
Hydrophobicity values were normalized between 0 and 1 on a per-protein basis.


\subsection{Half Sphere Amino Acid Composition}
\noindent
\emph{Number of features}: 40

\noindent
\emph{External tools}:  None

\noindent
\emph{Description}:
These features capture the residue composition in a neighborhood of the residue of interest.
The neighborhood is defined as all residues with at least one atom closer than 8\AA{} to at least one atom in the residue of interest.
An array of counts is created which stores the number of each type of residue in the neighborhood, which is then divided by the total count of all types.
This array is calculated for two regions, separated by the null space of vector that is the average of the unit vectors from the alpha carbon to the side chain atoms.


\section{Edge Features}

\subsection{Average Atomic Distance}
\noindent
\emph{Number of features}: 1

\noindent
\emph{External tools}: None

\noindent
\emph{Description}:
This feature is computed by taking the average of the distance between any two atoms of the two residues of interest.
The raw distance is then passed through a Gaussian function: 

\begin{equation}
f(x) = e^{-x^2 / \sigma^2},
\end{equation}

\noindent
where sigma is chosen by model selection, so that this feature has a higher value for residues that are closer together.


\subsection{$\text{C C}_{\alpha} \text{O}$ Angle}
\noindent
\emph{Number of features}: 1

\noindent
\emph{External tools}: None

\noindent
\emph{Description}:
This feature captures the relative orientation of two residues as defined by the normals to their adjacent peptide planes.
Let $X_{C_{\alpha}}$, $X_{C}$, and $X_{O}$ be the positions vectors of the alpha carbon, carboxyl-group carbon, and carboxyl-group oxygen respectively.
The normal is then defined as:
\begin{equation}
\hat{N} = \frac{(X_{O} - X_{C}) \times  (X_{C_\alpha} - X_{C})}{||(X_{O} - X_{C}) \times  (X_{C_\alpha} - X_{C})||}, 
\end{equation}

\noindent
where $\times$ denotes the cross product, and $ || \cdot || $ the $\textsc{L}^2$ norm.
The angle between residues $i$ and $j$ is then:
\begin{equation}
\text{C C}_{\alpha} \text{O} = \text{cos}^{-1}\bigg(\frac{\hat{N_i} \cdot \hat{N_j}}{|| \hat{N_i} ||~|| \hat{N_j} || } \bigg)
\end{equation}

\noindent
Finally, the angle is normalized to be between 0 and 1 by dividing by $2 \pi$. 


\chapter{Software Implementation}
\label{appendix:implementation}

The main software used in this thesis was written in Python version 2.7.13.
Rather than writing several individual scripts which each ran their own experiments, the software was written as a unified framework which would allow consistent use of data, rapid experimentation, and reuse of critical experimental code.
This experiment briefly describes some of the notable elements of the software.

\section{Experiment Specification Files}

Experiments are specified using the YAML data format, a human readable data serialization language. 
To run an experiment, the user calls the main experiment running script and passes the specification file name as an argument.
The specification file fully defines the experiment(s) being run, to include the data format, model architecture, and training/testing procedure. 
This has a number of advantages.
First, it ensures traceability from experimental conditions to results. 
A copy of the experiment file is placed in the same directory of the experiment results, ensuring that any results from past experiments can be understood and interpreted appropriately.
Second, it allows rapid iteration of experimentation, since running a modified version of an experiment is requires a simple modification of the specification file.
Third, it allows concurrent researchers to review each others' work without having lengthy discussions about the precise experimental settings used.
For illustration, the contents of an example YAML file are shown below (for Sum Coupling, receptive field size 21, for one, two , three, and four layers):


\begin{minted}[frame=single, fontsize=\small]{yaml}
type: 'experimentset'
common:
  type: 'traintest'
  args:
    test_frequency: 80
    vis_frequency: 80
    test_before_train: True
    num_epochs: 80
    minibatch_size: 128
    test_batch_size: 2000
    checkpoint_frequency: 80
  data:
    dataset: "dbd5"
    train_list: ["train_list_w_2B42.yml", "validate_list.yml"]
    test_list: "test_list_w_3R9A.yml"
    n_train: -1
    n_test: -1
    train_ratio: 10
    test_ratio: -1
    nans: "global_median"
    features:
    - "WINDOWED_POSITION_SPECIFIC_SCORING_MATRIX"
    - "RELATIVE_ACCESSIBLE_SURFACE_AREA"
    - "RESIDUE_DEPTH"
    - "PROTRUSION_INDEX"
    - "HALF_SPHERE_EXPOSURE"
    - "RES_DIST_AVE"
    #- "RES_DIST_STD"
    - "ANGLE_CCAO"
    #- "ANGLE_HSAA_COS"
    feature_args:
      profile_params:
        window_size: 1
        blast-database: "/s/<host>/a/tmp/blastdb/db"
      distance_matrix:
        preprocess: ["gaussian_kernel", 18]
      angle_matrix:
        preprocess: "divide_pi"
    representation: "pairwise_data_per_protein_with_hood_indices"
  metrics:
    - "learning_rate"
    - "params"
    - "loss_train"
    - "loss_test"
    - "roc_train"
    - "roc_test"
    - "rfpp_train"
    - "rfpp_test"
  model:
    name: "PWClassifier"
    layer_args: {init: "he", coupling: "sum", bias_init: "zero",
        nonlin: "relu", dropout_keep_prob: 0.5}
    loss: "tf_weighted_crossent"
    loss_args:
      pn_ratio: 0.1
    optimizer: "tf_sgd"
    optimizer_args:
      learning_rate: 0.1
experiments:
- - "1_layer_rf20"
  - data:
      nhood_size: 20
    model:
      layers:
      - ["star_v_conv", {filters: 256}]
      - ["merge_gather_swap_paired_examples", 
              {mode: ["concat"]}, ["merge"]]
      - ["dense", {out_dims: 512}]
      - ["dense", {out_dims: 1, nonlin: "linear"}]
      - ["combine_swapped"]
- - "2_layer_rf20"
  - data:
      nhood_size: 20
    model:
      layers:
      - ["star_v_conv", {filters: 256}]
      - ["star_v_conv", {filters: 512}]
      - ["merge_gather_swap_paired_examples", 
              {mode: ["concat"]}, ["merge"]]
      - ["dense", {out_dims: 512}]
      - ["dense", {out_dims: 1, nonlin: "linear"}]
      - ["combine_swapped"]
- - "3_layer_rf20"
  - data:
      nhood_size: 20
    model:
      layers:
      - ["star_v_conv", {filters: 256}]
      - ["star_v_conv", {filters: 256}]
      - ["star_v_conv", {filters: 512}]
      - ["merge_gather_swap_paired_examples", 
              {mode: ["concat"]}, ["merge"]]
      - ["dense", {out_dims: 512}]
      - ["dense", {out_dims: 1, nonlin: "linear"}]
      - ["combine_swapped"]
- - "4_layer_rf20"
  - data:
      nhood_size: 20
    model:
      layers:
      - ["star_v_conv", {filters: 256}]
      - ["star_v_conv", {filters: 256}]
      - ["star_v_conv", {filters: 512}]
      - ["star_v_conv", {filters: 512}]
      - ["merge_gather_swap_paired_examples", 
              {mode: ["concat"]}, ["merge"]]
      - ["dense", {out_dims: 512}]
      - ["dense", {out_dims: 1, nonlin: "linear"}]
      - ["combine_swapped"]
\end{minted}


\section{Convolution Functions}

Because all convolution approaches used a similar neural network architecture, each layer type was implemented as a function, and the relevant function is named in the experiment specification file (for example, "star\_v\_conv" in the YAML code above).
Each layer function would receive as inputs the outputs of the previous layer(s), and any number of keyword arguments specific to that layer.
The layer would define any TensorFlow operations on the inputs that produce the layer's output.
Below is an example function implementing a convolutional layer, which is a simplified version of "star\_v\_conv" mentioned above.

\begin{minted}[frame=single, fontsize=\small]{python}
def star_v_conv(layer_input, filters, init, bias_init, nonlin, 
                coupling, dropout_keep_prob, bias, **kwargs):
    # get inputs 
    vertices, edges, nh_indices = layer_input
    v_shape = vertices.get_shape()
    e_shape = edges.get_shape()
    nh_size = nh_indices.get_shape()[1].value

    # create weights
    Wvc = tf.Variable(initializer(init, (v_shape[1].value, filters)))
    bv = tf.Variable(initializer(bias_init, (filters,)), name="bv")
    Wvn = tf.Variable(initializer(init, (v_shape[1].value, filters)))
    We = tf.Variable(initializer(init, (e_shape[2].value, filters)))

    # create central vertex signals
    Zc = tf.matmul(vertices, Wvc, name="Zc")

    # create neighbor signals
    e_We = tf.tensordot(edges, We, axes=[[2], [0]], name="e_We")
    v_Wvn = tf.matmul(vertices, Wvn, name="v_Wvn")
    if coupling == "sum":
        Zn = tf.divide(
             tf.reduce_sum(tf.gather(v_Wvn, nh_indices), 1) \\
             + tf.reduce_sum(e_We, 1), \\
             nh_size)
    elif coupling == "product":
        Zn = tf.divide(
             tf.reduce_sum(tf.gather(v_Wvn, nh_indices) \\
             * e_We, axis=1), \\
             nh_size)

    # signal
    sig = Zn + Zc
    if bias:
        sig += bv

    # pass through the nonlinearity
    z = tf.reshape(nonlin(sig), tf.constant([-1, filters]))
    
    # dropout
    z = tf.nn.dropout(z, dropout_keep_prob)
    return z

\end{minted}



%%%%%%%%%%%%%%%%%%%%%%%%%%%%%%%%%%%%%%%%%%%%%%%%%%%%%%%%%%%%%%%%


%%%%%%%%%%%%%%%%%%%%%%%%%%%%%%%%%%%%%%%%%%%%%%%%%%%%%%%%%%%%%%%%
\end{document}
