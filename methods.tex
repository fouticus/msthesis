\chapter{Methods}
\label{chap:methods}

The methods presented in this paper were drawn from a desire to exploit the local information around residues when performing interface prediction. 
This is biologically motivated because the neighborhood around a residue impacts its propensity to interact.
It was noted in the introduction that convolutional neural networks are one way of exploiting local structure but are limited to regular grids. 
Since proteins are irregular 3D structures, there is no natural representation as a regular grid, we therefore need a more appropriate representation to which convolution can hopefully be adapted.
In this work we represent proteins as graphs.


\section{Proteins As Graphs}

A (undirected, unweighted) graph $G$ consists of a set of $n$ vertices, $V=\{v_1, v_2, ..., v_n\}$, and a set of $m$ edges, $E=\{e_1, e_2, ..., e_m\}$ where each edge is incident to two vertices.
In a protein graph, each vertex represents a residue in the protein, and each edge represents the relationship between two residues (in this case, two residues in the same protein, as opposed to two residues in different proteins which are possibly part of an interface).
Thus any information pertaining to a particular residue, in the form of a feature vector, can be associated with the relevant vertex.
The features used in this work are drawn from features used in prior interface prediction work \cite{minhas2014} and pertain to the sequence conservation and the structure of a particular residue. 
Likewise, any information pertaining to the relationship between two residues, in the form of a feature vector, can be associated with the relevant edge.
The edge features used here describe the distance between and relative orientation of two residues.
This relationship is defined between any two residues in the protein, so the graph is complete. 
A detailed explanation of each feature is contained in Appendix \ref{appendix:features}

This representation is a structural abstraction of the original protein to a well studied mathematical object.
It does not rely on a coordinate system as would be the case when working with raw 3D coordinates of residues.
This makes biological sense because proteins often have no natural orientation in the cell.
However, the graph retains 3D structural information in the form of the features contained on its edges, and can be thought as embedded in an underlying metric space. 
This property is useful when defining local neighborhoods around vertices, which is necessary when designing convolutions

\section{Graph Convolutions}
Though the formulation and application of graph convolutions presented in this paper are new, graph convolutions have existed in the literature for several years.


\subsection{Prior Work}
Recent years have seen increased attention for problems involving graph structured data, prompting developments in graph convolutions to perform various tasks on those data~\cite{bronstein2016}.
These approaches have generally fallen into two categories, \textit{spectral} and \textit{spatial}.

Spectral methods are based on linear functions in the "frequency domain" of a graph, defined using the laplacian operator $\mathcal{L}=I-D^{-1/2}WD^{-1/2}$, where $W$ is a similarity matrix (containing edge weights), and $D$ is a diagonal matrix containing the degree of each vertex~\cite{henaff2015}.
%TODO: add other citations of spectral methods?
Each filter in a spectral convolution implies a weighting of each frequency in the spectral decomposition of the graph~\cite{mallat2009}.

Spatial methods instead define operations in a localized neighborhood of a vertex~\cite{henaff2015}.
%TODO: add other citations of spatial methods?
%TODO: mention(?): scaling to large graphs difficult because factoring large matrices difficult
Each neighborhood constitutes a receptive field where a convolution operation is performed. 
Various convolution operations are defined in the literature(CITE), which commonly involve a vector of weights and take a weighted sum of neighbors, much like a discrete convolution on a grid can be viewed as taking a linear sum of grid elements within the receptive field.
%TODO: cite papers with different neighborhood convolutions (write more about them?)
Spatial convolutions are more directly analogous to grid based convolutions as described in the introduction, but introduce an problem of correspondence when translated to graphs.

\subsection{The Problem of Receptive Field Correspondence}
When convolving on a grid, each receptive field has identical structure (for example 3x3 pixels in an image), so there is a defined correspondence between receptive fields, such  that the same weights are applied to corresponding portions of all receptive fields. 
For example, the upper left pixel in a 3x3 receptive field is always multiplied by the same weight when taking the weighted sum, regardless of which receptive field is being considered.
With graphs, there is often not such a correspondence from one receptive field to another (there is no "upper left" vertex in a neighborhood around a central vertex common to all receptive fields), and in fact the number of neighbors in a receptive field is not even guaranteed to be constant.
This problem of correspondence has been addressed in various ways which are summarized below.
\begin{enumerate}
	\item \textit{Imposed ordering of neighbors}. This approach generates a correspondence between two receptive fields by ordering the neighbors in each and associating neighbors of a common position. 
	Ordering methods are heuristic and rely on some understanding of the problem domain.
	%TODO: research ordering methods and cite them
	They also typically require the number of neighbors in a receptive field to remain the same.
	This approach allows filter weights which are coupled to a particular position in the ordering, which hopefully has some invariant significance in a receptive field.
	
	\item \textit{Order free treatment of neighbors}. This approach ignores the need to establish a correspondence between receptive fields and instead treats all neighbors in the same way.
	For example, rather than apply different weights to neighbors depending on their position in an ordering, the same weights are applied to each neighbor.
	This allows for different sizes of receptive fields and avoids choosing an ordering method, but lacks the ability to treat different neighbors differently based on their relationships to each other and to the central vertex.
	
\end{enumerate}


\subsection{Order Free Coupled Graph Convolutions}
This paper presents a structural convolution which avoids imposing an ordering on the neighbors in a receptive field but avoids treating all neighbors the same.
This is accomplished by incorporating information from the edges between each neighbor and the central vertex.
We present two forms of graph convolution which differ in how the edge information is incorporated, denoted \textit{sum coupling} and \textit{product coupling} respectively.
For a given vertex $i$ on the graph and a local neighborhood of vertices $\mathcal{N}$, we define the output of a sum coupled graph convolution to be
\begin{equation}
h_i = \sigma \bigg( W^{\textsc{c}} x_i + \frac{1}{|\mathcal{N}_i|}\sum_{j \in \mathcal{N}_i} W^{\textsc{n}} x_j + \frac{1}{|\mathcal{N}_i|}\sum_{j \in \mathcal{N}_i} W^{\textsc{e}} A_{ij} + b \bigg),
\label{eq:sum_coupling}
\end{equation}
where $x_i$ is the feature vector associated with vertex $i$, $A_{ij}$ is the feature vector associated with edge $(i, j)$, $W^c$, $W^N$ and $W^E$ are weight matrices, and b is a vector of biases. 
If there are $l$ vertex channels, $p$ edge channels, and $k$ filters, then $W^c$ and $W^N$ have shape $(k, l)$, $W^E$ has shape $(k, p)$, and $b$ has shape $(p, 1)$
Intuitively, this calculates an activation for the central vertex, each neighbor vertex, and each edge between a neighbor and the central vertex.

Even thought the same weight matrix is being applied to every neighbor, we incorporate information about how each neighbor relates to the central vertex, allowing differentiation between neighbors. 

This formulation loses the association between a neighbor and the edge which connects it to the central vertex. 
Therefore we also define product coupled graph convolution to be 
\begin{equation}
z_i = \sigma \bigg( W^{\textsc{c}} x_i + \frac{1}{|\mathcal{N}_i|}\sum_{j \in \mathcal{N}_i} (W^{\textsc{n}} x_j) \odot (W^{\textsc{e}} A_{ij}) + b \bigg),
\label{eq:prod_coupling}
\end{equation}
Where $odot$ denotes the elementwise product between two vectors or matrices. 
In this formulation, each vertex signal is multiplied by the signal of the edge which connects it to the central vertex.
In one interpretation, this allows a neighbor's influence on the overall activation to be modulated according to its relationship to the central vertex (embodied in the edge). 
For protein graphs, this means neighboring residues will contribute more or less to the activation depending on their distance from and relative orientation to the central vertex.

Both sum coupled and product coupled graph convolutions are invariant to rotations or translations in space, don't impose an ordering in the neighbors or a correspondence between receptive fields of any kind, allow for different numbers of neighbors, and account for the different relationships between neighbors and the central vertex. 
The receptive fields are always defined around a central vertex, so the results of convolution can be applied to that vertex.
This retains the graph structure after each convolution, so convolutional layers are stackable, as with images.

A note on receptive fields: 
The graphs used in this paper are complete and embedded in a metric space, so a receptive field can be defined using a threshold $\delta>0$ such that all vertices within the threshold are included in the receptive field.
All neighbors are guaranteed to share an edge with the central vertex.
For incomplete graphs, a receptive field can be defined as all vertices within $k$ hops of the central vertex. 
If $k=1$, both versions of graph convolution can directly be applied.
If $k>1$, then there are multiple ways of applying product coupled graph convolution, though it is not clear which would be more appropriate and/or effective in any particular application.
%TODO: say more?

These graph convolutions allow the detection of local patterns on a single graph, and produces a new representation at each vertex.
Partner specific protein interaction, however, requires classifying pairs of residues (vertices in different graphs), so a pairwise neural network architecture must be used. 

\section{Pairwise Deep Learning Architecture}
To make predictions on pairs of residues from different proteins, we construct a pairwise architecture comprised of three main sections: the \textit{pre-merge} (or \textit{leg} layers), the \textit{merge} layer, and the \textit{post-merge} layers. 
Even though predictions are being made for pairs of individual residues, convolution depends on surrounding residues, or in the case of stacked convolutional layers, the generated representation of surrounding residues.
Therefore it is less redundant to convolve an entire protein at once than recreate convolutional representations for each pair of residues separately. 
Therefore the input to a pairwise classifier neural network is a pair of protein graphs derived from the \textit{ligand} and \textit{receptor} in a known protein complex.
Each protein is convolved individually in a separate leg, where the number of convolutional layers in each leg is the same and model parameters and other weights are shared across the legs.

The merge layer then combines the representations of residues from one graph with the representations of residue from the other into pairs. 
In theory, this merge process should be symmetric because the network should perform identically regardless of which protein is used as input into which leg.
For example, the elementwise sum, elementwise product, and outer-product are all commutative and therefore produce symmetric output.
Another option would be to combine pairs in a non symmetric way, such as concatenating the two representations together, but then averaging the predictions between each ordering of the representations.
This would also produce a symmetric prediction function.
%TODO: talk about pairwise kernels?

After merging, the resultant combined representation for each pair of residues can be passed through a number of dense layers until reaching the last dense layer, which has a single output.
The output can be compared an encoded label indicating whether (+1) or not (-1) the pair constitute part of the true interface. 

%TODO: figure...

